\section{Methodology}
\label{sec:methodology}

In the following, we define capacity as the access link capacity offered by the 
ISP to the user. In the case of the Comcast dataset, capacity is 105 Mbps for 
the \control set and 250 Mbps for the \test set. We define usage as 
the actual traffic (bytes uploaded or downloaded) by the household being 
studied. Similarly, the average data rate is derived by calculating the 
usage in a 15 min time slot (in kbps). We also refer to this as utilization,
and compare it to the access link capacity in our analysis.

\subsection{Different Perspectives of Utilization}

The FCC has the responsibility to increase the availability and deployment of broadband
throughout the US (with the broadband threshold benchmark defined as 25 Mbps in downlink and
3 Mbps in uplink). Their progress report states that: given the option, users will adopt a
higher tier bandwidth~\cite{}, thereby meriting the high investment. However, a survey
conducted by NCTA showed that the largest deterrent to broadband adoption is that users
do not \emph{need} broadband (the second largest is the cost). The conflicting view of 
the ISP is that the cost of deployment in an unchartered area is too high, unless
a significant number of households \emph{need} it. Thus, both parties are asking the same
question: do people \emph{need} a higher capacity, i.e., what is their \emph{utilization}
as compared to the capacity?

Previous research shows that the utilization will increase as the capacity of the access
link increases~\cite{}, and also that utilization\footnote{called demand in their work} and capacity
follow a law of diminishing returns~\cite{bischof2014broadband-behavior}. However, such
studies have been biased by studying users that actually required a higher capacity
for their usage. It is inevitable that such a correlation would exist for 
such households, whose utilization is bottlenecked by the ISP.

In this work, we ask a more fundamental question: \emph{how much does the user behavior
change with increasing capacity}. Specifically, when the capacity is already very high
and the user has not opted for an increase, does their utilization still vary with capacity?
Both the FCC and the ISPs have a different perspective of the utilization:

% the overall usage is very similar
% max peak/day are the same, but on the lower end peak usage per device changes
% much lesser than the capacity but it is still a change
\paragraph{The FCC perspective: }\emph{Utilization as adoption} of a higher capacity
link when available (but not under the constraints of a much higher cost). The answer to this
question is important to the FCC to encourage further deployment of high tier links throughout
the US. Essentially, if \emph{any} change is observed in link utilization due to the upgrade in our dataset, the FCC may interpret that as \emph{adoption} to the higher available tier.

% The max util DOES NOT change => we are not the bottleneck and users are satisfied.
\paragraph{The ISP perspective: }\emph{Utilization as a capacity bottleneck}, i.e., if
the ISP can show that the \emph{maximum utilization} of a household does not vary
with increasing capacity, it will prove there is not enough demand to offer a higher tier.  
The ISP needs the answer to this question for future capacity planning, and the cost-analysis
for the investment of new technology in any area.
For example, Google Fiber is now expanding to Salt Lake City, from where we received our dataset.
The analysis of change in user behavior with capacity will estimate the number of users that
actually \emph{need} the higher capacity service offered by Google.

Thus there is a need to measure \emph{utilization} at times when users \emph{need} the Internet
capacity the most: the peak usage.

\subsection{Importance of Measuring Peak Usage}

Internet usage throughout a day follows diurnal sleep-patterns, and researchers
have shown that such patterns are in fact correlated with GDP, Internet allocations,
as well as electrical consumption of a region~\cite{ant-diurnal-web}. This makes
the study of usage behavior extremely relevant to the governmental bodies responsible
for development, such as the FCC, when considering policy decisions. 

\paragraph{Prime Time: }The daily diurnal nature of usage patterns across many
households naturally requires the provider to design networks capable of handling 
load at the peak times in a day. Such peak times are usually observed during
evening hours, and the data transferred at this time is called peak usage.
The FCC defines \emph{Prime Time} as the local time from 7:00 PM to 11:00
PM~\cite{fcc2014measuring-broadband}, when many
households heavily consume real-time entertainment traffic (video), seen as primarily
responsible for high usage during these hours. Latency and performance are adversely
affected during prime-time, causing bottlenecks at home, the last mile, in
transit, or at the content server. For example, the Sandvine Global
Internet Phenomena Report \footnote{The Sandvine Reports ~\cite{sandvine2014report1,
sandvine2014report2}are released bi-annually and
contain a detailed analysis of aggregate Internet usage. They are also referred
to in the FCC reports~\cite{fcc2015progress-report, fcc2014measuring-broadband,
fcc2014progress-report}} showed that devices in the same household selected Netflix's
own CDN (OpenConnect) during off-peak hours, and third party CDNs (with differing
performance) during prime-time. This may happen because Netflix OpenConnect is over-utilized
during prime time~\cite{sandvine2014report1}.

\paragraph{Prime Time Ratio: }To measure the concentration of network usage during prime time,
Sandvine defined the \emph{Prime-Time ratio} as the ``absolute levels of network traffic
during an average peak period hour with an average off-peak hour''. Based on the FCC
definition of prime-time hours (7p-11p), we measure the daily prime-time ratio of $set_{full}$
in section~\ref{subsec:primetime}.

\paragraph{Peak Ratio: }The Sandvine Reports show that although the mean usage has remained
stable for the past few years, usage during peak-times has increased
drastically~\cite{sandvine2014report1}. To measure this growth, they introduce the
concept of peak period, measured when the network is within 95\% of its highest point.
Although, these reports present a good view into aggregate usage patterns over a month,
they neglect to analyze usage characteristics individually. Inspired by their
definition, we measure the disparity between the 90 percentile of the peak and median
usage of each household within a day, and call this the \emph{Peak-Ratio}. In
section~\ref{subsec:peakratio} we show that the peak ratio can be used
to divide users in the same tier based on their usage behavior.

In the next section, we analyze spacio-temporal network usage behavior:
\textbf{Time Series Behavior (TS): }aggregating the usage (and utilization) per household
over time (daily or weekly, per time slot). \textbf{Distribution Across Devices (CDF): }aggregating over time slots per day to measure utilization per device. We use the prime-time ratio and peak usage
as criteria to evaluate usage behavior and interpret utilization.