\section{Introduction}

Many networks require cooperation among multiple independently operated
and often mutually distrustful components or administrative domains.
For example, the Internet comprises tens of thousands of independently
operated networks; enterprise networks require the cooperation of
multiple administrative or organizational units; and a multi-tenant data
center inherently involves multiple administrative entities.  In these
settings, each independently operated entity should be able to specify,
reason about, and enforce permissions on the actions that other entities can
perform on network traffic.

Unfortunately, today's networks have no built-in mechanisms to encode
the authorization of network entities that try to perform actions on
network traffic. Networks today solve these problems with one-off
approaches that are tied to specific protocols and mechanisms that are
difficult to deploy and reason about.  For example, the Internet's
interdomain routing protocol, the Border Gateway Protocol~(BGP), allows
independently operated networks (Autonomous Systems, or ASes) to
exchange routes, but BGP itself provides no mechanism for allowing the
owner of an IP prefix to specify who should be allowed to announce that
prefix. Similarly, enterprise network operators have no means of
specifying the authorization that organizations within the larger
enterprise should have (\eg, whether and how a certain organization
should be able to configure a shared firewall).  Even the mechanisms
that do exist for this purpose (\eg, Virtual LANs, RPKI) are coarse and
easily misconfigured. Network operators often use virtualization to
isolate portions of a network, but isolation is binary (\ie, either two
parts of the network are isolated or they are not), whereas
authorization policies are often more fine-grained and need not require
complete isolation.

We need to rethink how authorization occurs in communications networks,
and now is the time to do so.  The need for better approaches is clear:
security problems in networks are frequent and increasingly serious, and
existing approaches are not working.  Yet, we believe that better, more
principled solutions are now within reach, thanks to software defined
networking (SDN), which allows us to think about authorization of
network control independently of the mechanisms of configuration and
forwarding.  Our design is inspired by a clean separation between the
network's control and data planes, which technically exists in
conventional networks but for which software defined networking (SDN)
provides much cleaner abstractions. Thus, our approach is general and
can apply to both conventional networks and to SDN.

A typical network configuration or control application (the ``control
plane'') issues commands via some control channel that ultimately
translate to {\em actions} on traffic flows. In this paper, we develop
(1)~an authorization logic that determines whether any particular
requested action is authorized; and (2)~a mechanism for implementing
that logic.  We introduce a {\em reference monitor} that can determine
whether a principal is authorized to perform actions on {\em network
  flowspace} (\ie, groups of traffic flows that have header values in
common), as shown in Figure~\ref{fig:reference-monitor}. This reference
monitor is analogous to the ones enforcing authorization for certain
operating system actions~\cite{taos}. 

To implement the logic that we develop, we introduce a {\em reference
  monitor} that determines whether the {\em principal} (\ie, any entity
that is trying to perform an action, such as a network administrator
acting on behalf of a network) attempting to perform a write action to
the data plane or a read action on network traffic is authorized to do
so.  In the context of a network, the principal is any control
application, and an action is a forwarding rule that could be state
installed into network device that affects how traffic is forwarded
(\eg, an entry in a match-action table).  The reference monitor takes
both authorization policies (\eg, one network principal's statement that
a different principal is authorized to perform certain actions) and
credentials as input (\eg, information about who owns an IP prefix) and
attempts to prove that the principal who issues a request is authorized
to perform the requested actions on the data plane.  The reference
monitor resides in the network's control plane, in between the network
configuration (or control program) and the subsystem that ``compiles''
this configuration to the network's routers and switches, as shown in
Figure~\ref{fig:reference-monitor}.  For the purposes of this
paper---and as is consistent with previous work on authorization logics
in the systems community---we assume that the reference monitor is
trusted.

To allow a reference monitor to make decisions about authorization, we
develop \name{}, a formal logic for authorizing network control,
implement it in an SDN controller, and evaluate it in multiple network
settings, including enterprise networks and interdomain
routing.  %Using examples from
%real-world network configuration scenarios, 
We demonstrate that \name{} is both more expressive and more flexible
than existing solutions to authorization (\ie, BGPSEC for interdomain
routing).  Our prototype implementation and evaluation shows that
\name{} scales to real-world deployments; all of the code for \name{}
and the experiments that we have conducted are publicly available.

Realizing \name{} presents several research challenges.  First, although
existing authorization logics provide a useful foundation, they
typically concern operations on discrete operating system objects (\eg,
files, processes) and thus do not apply to network flowspace, which has
more complicated relationships, such as subset relationships.  Our work
extends authorization logics from previous work so that they can be
applied to network control. In a network setting, an object is a set of
data packets that can be represented as a subset of flowspace; a request
to perform an operation is an attempt to read (\ie, see a particular set
of packets) or write (\ie, affect the packets' properties or how they
are forwarded); and a principal is a network control policy that is
implemented either as device configuration or, in the case of SDN, a
control program.
Second, we must design the reference monitor to
operate scalably in an operational network; because
\name{} involves set operations over network flowspace as opposed to
finite operations on discrete objects, designing the implementation to
scale with the number of participants and policies was challenging.


\begin{figure}[t!]
  \centering
\includegraphics[width=\linewidth]{figures/reference-monitor-2}
\caption{A reference monitor between the network's control plane and
  data plane determines whether a network control program (speaking for
  a principal) is authorized to perform read or write operations. The
  reference monitor could be coupled with an SDN controller or a network
configuration management system.}
\label{fig:reference-monitor}
\end{figure}

We present three contributions: (1)~we extend existing
state-of-the art authorization logics (\eg, NAL~\cite{nexus}) to allow
reasoning about actions on network flowspace; (2)~we design an overall
architecture and accompanying algorithms to ensure that the logic
operates correctly and efficiently for real-world use cases; (3)~we
apply the logic and its implementation to several real world examples to
demonstrate the logics expressiveness and efficiency.  Our application
of \name{} to a variety of network types and authorization scenarios
demonstrates that the logic is expressive enough to apply to many
network settings.  Our evaluation demonstrates that \name{} is efficient
enough to authorize (or disallow) requests from network principals in
real time.

The rest of the paper is organized as follows.
Section~\ref{sec:motivation} motivates the need for an authorization
logic in today's networks. Section~\ref{sec:background} provides an overview of authorization
logics and how they are commonly applied to systems; 
Section~\ref{sec:logic} describes
the logic and how it can be used to verify authorization;
Section~\ref{sec:implementation} describes our prototype implementation
of \name{} in an SDN controller.  Sections~\ref{sec:enterprise}
and~\ref{sec:bgp} demonstrate applications of \name{} to enterprise
networks and backbone networks, respectively. Section~\ref{sec:related}
presents related work, and Section~\ref{sec:conclusion} concludes.