\subsection{Prime Time Ratio}
\label{subsec:primetime}

To characterize the prime-time ratio, as defined in ~\ref{sec:methodology}, we
calculate the aggregate data transferred at the ISP in an average prime-time hour,
and divide it by the off-peak average.

Prompted by the monotonically increasing trend of usage behavior during daytime hours on
weekdays (figure~\ref{fig:TS-data-rate-daily})
we calculated the prime-time ratio for each four hour period throughout the day
to find the evening hours with the largest ratio.
In our dataset, the prime time ratio peaks at 8:00 PM -- 12:00 AM,
rather than FCC's definition of 7:00 PM -- 11:00 PM. This discrepancy could be limited
only to the high tier households in our dataset, but we deem that unlikely.
%as the demand for traffic during prime time should be independent of the tier.
Another reason could be that prime time is delayed globally with the rise in real
time entertainment's contribution to traffic.

\begin{figure}[ht!]
\begin{minipage}{\linewidth}
\centering
\includegraphics[width=\linewidth]{figures/prime-time-ratio-by-date[replace].png}
\caption{Prime Time ratio showing weekly pattern + differences during holiday periods (Thanksgiving, Christmas)}
%http://riverside.noise.gatech.edu:8083/separated/full/prime-time-ratio-by-date.png
\label{fig:TS-prime-time-ratio}
\end{minipage}
\end{figure}

We use our updated definition of Prime Time (table~\ref{tab:eval-criteria}) to calculate and plot
the Prime Time ratio per day for the \test and \control sets in
figure~\ref{fig:TS-prime-time-ratio}. A comparison shows that
the \test set's prime time ratio is 10\% lesser, supporting the observation from
section~\ref{subsec:behavior} that showed that the usage during prime time is similar
across both sets, but the usage in off peak hours is higher in the \test set.

\todo{see todo.txt for remaining analysis and concluding remarks}