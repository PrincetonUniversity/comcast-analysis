% CONTRIBUTIONS paragraph instead of ISSUES as part of introduction
% follow directly from the results section
Our contributions in this work are as follows:
\begin{itemize}
\itemsep0em
%%%%%%%%%%%%%% usage behavior
%\item We characterize \textbf{usage behaviors} and show that aggregate network usage seen by the ISP does not show a trough in the middle of the day for high capacity tiers.
%, as observed by aggregate usage across all tiers. % This motivates the need to study usage per tier separately as usage behavior differs with tiers.
%%%%%%%%%%%%%% prime time
\item We reaffirm the importance of measuring performance during the prime-time peak hours, and urge the FCC to revisit and standardize the measurement and interpretation of \textbf{prime-time ratio}. 
%we show through our analysis that the \textbf{prime-time ratio
%(ratio of average data rate during prime time hours to non prime time hours)
%does not follow the conventional definition stated by the FCC.
%%%%%%%%%%%%% different perspective of utilization
\item We shed light on the user's and ISP's \textbf{differing perspectives of \emph{capacity utilization}}, and show that indeed, the user's behavior does change when offered a higher capacity link even though the overall utilization stops increasing after a certain upper limit.
%%%%%%%%%%%%% user taxonomy and multiple benchmarks
\item We explore a user taxonomy based on \textbf{varying usage behavior within the same tier}, and suggest that the FCC adopt multiple benchmarks to better characterize broadband availability, deployment, and adoption in the US.
%We suggest that the FCC adopt multiple benchmarks to better characterize broadband availability, deployment, and adoption in the US, based on our analysis of varying usage patterns within the same ISP tier.
%We explore a taxonomy of different users based on varying traffic patterns of users within the same capacity tier, and 
\end{itemize}



%In particular, we believe that there is a need to: (a) define benchmarks based on standards other 
%than broadband speed, such as usage, (b) define multiple benchmarks rather than aggregates based on
%type of usage, and (c) re-examine peak time usage with recent changes in usage patterns to estimate
%for capacity planning. We also realize that the \FCC requests comments on the issues of broadband
%benchmarking [section ~\ref{sec:issues}] in future. To this aim, we use a dataset collected purposefully
%to validate and improve the \FCC policy on broadband benchmark.