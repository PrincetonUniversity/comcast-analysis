\section{Background and Related Work}\label{sec:related}

1. previous academic studies of usage - study low tier and natural experiments. Our approach is highly controlled experiments

2. industrial trends like sandvine and the FCC studies of usage and reports
\cite{fcc2015broadband-report}

3. Last para: no one 
Broadband analysis has recently attracted much attention
from the research community and the general public given
its important business and policy implications. A number
of efforts have focused on characterizing the availability and
performance of broadband services around the world [1, 2,
5, 12, 20, 28, 31, 33]. The focus of our work is on exploring
broadband services in their broader context, evaluating the
complex interplay between broadband service characteristics,
their market features and user demand.
Different aspects of the complex interplay between user
behavior, network services and operation has been explore
in previous work. Some recent studies have examined the
relationship between user behavior, network services and
the providers. In Dobrian et al. [13] the authors show that
poor connection quality can have a negative impact on a
user’s quality of experience. Blackburn et al. [3] study how
user behavior affects the economics of cellular operators.
Chetty et al. [7] perform a user study to understand the
effects of usage caps on broadband use. Other efforts
have explored additional factors that may influence service
demand, including the weather [6], service capacity [36] and
the type of region [8].



(instead of this say we did it)
The difficulty or outright impossibility of conducting controlled,
randomized experiments of user behavior at Internet
scale has been pointed out before. In his SIGCOMM 2011
Award presentation, Vern Paxson pointed to this issue
and suggested the use of natural experiments to explore
potential causal relationships with observational data. In
a recent paper, Krishnan and Sitaraman [21] explore the
use of related quasi-experimental design (QED) to evaluate
the impact of video stream quality on viewer behavior and
Oktay et al. [24] relies on it for causal analysis of user
behavior in social media. We opted for natural experiments,
rather than QED, as we consider the control and treatment
groups to be sufficiently similar to random assignment.