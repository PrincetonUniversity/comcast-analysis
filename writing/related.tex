\section{Related Work}\label{sec:related}

The measurement community has produced a plethora of studies of
broadband performance analysis, yet has performed relatively fewer
studies of traffic demand in broadband access networks.
The increasing availability of high-bandwidth Internet services and the
FCC's recent interest in exploring traffic  
demand as a broadband benchmark \cite{fcc2015progress-report} now calls
for increased attention to the relationship between user traffic demand
and broadband capacity.
% basically availability includes performance, price, usage caps, etc. etc.

Our work complements an earlier study by Bischof \ea \cite{dasu-imc2014},
who used {\em natural} experiments to investigate causal relationships between  
the traffic demand (which they refer to as ``user demand'', or ``usage''
in their paper) 
and factors such as service capacity, performance, and price. Bischof \ea showed 
that demand increases with capacity, but ``follows a law of diminishing  
returns''; in other words, increases in capacity for an already high
tier results in a lower 
increase in demand.
%, than if the upgrade would have occurred for a lower tier. 
Our work presents complementart results from a large-scale {\em
  controlled} experiment and examines in particular a high service tier
(105 Mbps) that has not been studied before. Our dataset mitigates the
affect of price, performance, and other potential biases (such as
regional \cite{dasu-weather, dasu-region}, capped usage
\cite{youre-capped}, and ``geek-effect'' \cite{dasu-imc2014}) by
limiting the dataset to a large number of users selected randomly from
the same service tier and location.

Zheleva \ea present a case study of the effects of an Internet service
upgrade, from 256~kbps satellite to 2~Mbps terrestrial wireless, in
rural Zambia~\cite{zheleva2013}.  This work observed that the stark change in traffic
demand three months after the upgrade caused a performance
bottleneck. In constrast, our case study focuses on traffic demands of
subscribers from much higher service tiers who are not continuously
bottlenecked by their access link; additionally, we study how users
adjust their traffic demands without informing them of the upgrade, thus
eliminating potential cognitive bias.


Other efforts such as \cite{maier-imc2012} study the characteristics of 
residential DSL broadband, and report the contributions of the most popular 
web applications to the total usage.
%They also note that most DSL users do not utilize their service capacity.
The bi-annual Sandvine reports \cite{sandvine20141h, 
sandvine20142h} provide an overview of overall Internet traffic 
demand from fixed lines and mobile carriers as well as an updated analysis of 
the most popular Internet applications. They showed that video accounts for 63\%
of traffic usage overall, and traffic demand peaks during the peak evening 
hours, possibly due to increasing video content consumption. Our work does not 
concern with the applications responsible for most traffic, but only with the 
peak period during which an individual subscriber's traffic demand is high.

%DASU: Other efforts have explored additional factors that may influence 
%service demand, including the weather [6], service capacity [36] and
%the type of region [8].
