\section{Background and Related Work}\label{sec:related}

Traditionally, broadband analysis has attracted much attention of the 
measurement community in terms of performance and economics, due to its policy 
implications. However with increasing availability of high bandwidth Internet 
services and FCC's recent interest in exploring traffic demand as a broadband 
benchmark \cite{fcc2015progress-report}, the focus has moved to evaluating the 
interplay between broadband demand and \todo{availability.}
% basically availability includes performance, price, usage caps, etc. etc.

Our work build upon earlier analysis of the 
relationship between traffic demand and service capacity by Bischof \ea 
\cite{dasu-imc2014}. In this prior work, natural experiments were used 
hypothesize and infer causal relationships between the traffic 
demand\footnote{referred to as user demand, or usage in their work} and service 
capacity, performance, or price. They showed that, accounting for price and 
performance, demand increases with capacity, but ``follows a law of diminishing 
returns'', \ie for a higher service tier the correlation between capacity and 
demand is lesser than that observed for a lower service tier. Our work 
complements their study and examines in particular a very high service tier (105 
Mbps) that has not been studied in depth before. Our dataset mitigates the 
affect of price, performance, and other unknown biases (such as regional 
\cite{weather, region}, capped usage \cite{}, and ``geek-bias'' \cite{}) by 
using a controlled experiment for a large set of users selected randomly.

Zheleva \ea present a case study of the affect of an Internet service 
upgrade in rural Zambia on Internet usage (and performance). This work 
concentrates on users who were already bottlenecked at a service capacity 
of 256 kbps on a satellite link before the upgrade, and compares it to the 
change in traffic demand after they were upgraded to a 2 Mbps terrestrial 
wireless link. They observe a stark change in usage behavior three months 
after the upgrade. Our work focuses on higher service tier subscribers, who are 
presumably not bottlenecked, and studies changes in traffic demand without 
informing users that they had been upgraded.

Other efforts such as \cite{imc102-maier} study the characteristics of 
residential DSL broadband in terms of the most popular applications and their 
contribution to the total usage.
%They also note that most DSL users do not utilize their service capacity.
The bi-annual Sandvine reports \cite{sandvine2014report1h, 
sandvine2014report2h} provide an overview of overall Internet traffic 
demand from fixed lines and mobile carriers as well as an updated analysis of 
the most popular Internet applications. They showed that video accounts for 63\%
of traffic usage overall, and traffic demand peaks during the peak evening 
hours, possibly due to increasing video content consumption.

%DASU: Other efforts have explored additional factors that may influence 
%service demand, including the weather [6], service capacity [36] and
%the type of region [8].
