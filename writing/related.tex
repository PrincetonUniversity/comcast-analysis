\section{Background and Related Work}\label{sec:related}

Traditionally, broadband performance analysis has attracted the attention of 
the measurement community. However with increasing availability of high 
bandwidth Internet services and FCC's recent interest in exploring traffic 
demand as a broadband benchmark \cite{fcc2015progress-report}, the focus has 
moved to evaluating the complex interplay between broadband demand and 
availability.
% basically availability includes performance, price, usage caps, etc. etc.

Our work build upon earlier study by Bischof \ea \cite{dasu-imc2014}.
Bischof \ea used natural experiments to investigate causal relationships between 
the traffic demand\footnote{referred to as user demand, or usage in their work} 
and factors such as service capacity, performance, and price. They showed 
that demand increases with capacity, but ``follows a law of diminishing  
returns'', \ie increases in capacity for an already high tier causes a lower 
increase in demand.
%, than if the upgrade would have occurred for a lower tier. 
Our work complements their study via a large-scale controlled experiment and 
examines in particular a high service tier (105 Mbps) that has not been studied 
before. Our dataset mitigates the affect of price, performance, and other 
potential biases (such as regional \cite{dasu-weather, dasu-region}, capped 
usage \cite{youre-capped}, and ``geek-effect'' \cite{dasu-imc2014}) by limiting 
the dataset to a large number of users selected randomly from the same service 
tier and location.

Zheleva \ea present a case study of the affect of an Internet service 
upgrade, from 256 kbps satellite to 2 Mbps terrestrial wireless, in rural 
Zambia. 
This work observed that the stark change in traffic demand three months after 
the upgrade caused a performance bottleneck. Our work focuses on higher 
service tier subscribers, who are presumably not bottlenecked, and studies 
changes in traffic demand without informing users of the upgrade.

Other efforts such as \cite{maier-imc2012} study the characteristics of 
residential DSL broadband, and report the contributions of the most popular 
web applications to the total usage.
%They also note that most DSL users do not utilize their service capacity.
The bi-annual Sandvine reports \cite{sandvine20141h, 
sandvine20142h} provide an overview of overall Internet traffic 
demand from fixed lines and mobile carriers as well as an updated analysis of 
the most popular Internet applications. They showed that video accounts for 63\%
of traffic usage overall, and traffic demand peaks during the peak evening 
hours, possibly due to increasing video content consumption. Our work does not 
concern with the applications responsible for most traffic, but only with the 
peak period during which an individual subscriber's traffic demand is high.

%DASU: Other efforts have explored additional factors that may influence 
%service demand, including the weather [6], service capacity [36] and
%the type of region [8].
