\section{Discussion}\label{sec:discussion}

Our results show that, for high-speed service tiers, the 
prime-time ratio does not change when upgrading from a lower tier to a higher 
one. The ISP may interpret this as no (insignificant) change in demand during 
the prime-time, when the network is expected to be most clogged. We
believe that given the opportunity, a service provider considering increase in 
capacity will not invest in offering a service upgrade in such a scenario. The 
ISP perspective is that the highest demanding subscribers will not be utilizing 
the higher capacity during the prime-time, thus the total demand will not 
increase. This question is considered by ISPs when planning capacity upgrades 
in the future, or considering investment in a new technology or region. For 
example, Google Fiber is now expanding to Salt Lake City, from where 
we received our dataset. The analysis of change in user behavior with capacity 
estimates that a low number of users already on the 105 Mbps service tier will 
actually increase their demand beyond the 105 Mbps capacity if they 
migrate to a higher capacity service offered by Google. 

In contrast, although our results show that the change in demand during 
prime-time was insignificant, we observe a significant increase in demand 
during non-prime-time hours. Furthermore, the increase in demand was higher for 
users who contribute the lowest fraction of traffic to the ISP. This may 
convince subscribers and policy makers alike that individually, the demand of a 
household is affected by the increase in access link capacity, especially for 
subscribers who did not have much demands. We believe that this is the 
perspective the FCC takes when considering deployment and adoption of 
broadband services. Essentially, if any change is observed in demand due to an 
upgraded service, the FCC may interpret that as \emph{adoption} of the higher 
available tier.

The complex relationship between service capacity and traffic demand motivates 
a further study, using controlled and natural experiments. As independent 
researchers in the measurement community, we believe it is our responsibility 
to 
provide a much required input to the FCC when it comes to increasing broadband 
adoption and defining a traffic demand based benchmark. We believe that our work 
is
directly useful to the Commission as a step forward in understanding Internet 
usage.