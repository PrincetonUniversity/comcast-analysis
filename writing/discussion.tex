\section{Discussion}\label{sec:discussion}

Our analysis of the randomized controlled experiment shows that, for high-speed
service tiers, an announced upgrade increases the demand. However, an increase
greater than 20~MB is observed 40\% of the subscribers. These subscribers
had peak-demands lower than, or comparable to, the median peak demand 
for the lower and higher tier groups. This implies that as the service tier
changed, the 95th percentile traffic demands of subscribers who contribute the
least to the total traffic volumes increases.

Our results also show that although the average subscriber demand is higher on the
weekend, the increase in demand is 20\% on weekdays during non-prime-time
hours. The prime time ratio decreases from 2.16 in the lower capacity service tier
to 1.88 in the higher capacity tier.



POLICY IMPLICATIONS:

Demand increases for lower bandwidth users consistently
Relatively more data is sent in off peak hours 


A service provider considering increase in 
capacity will not invest in offering a service upgrade in such a scenario. The 
ISP perspective is that the highest demanding subscribers will not be utilizing 
the higher capacity during the prime-time, thus the total demand will not 
increase. This question is considered by ISPs when planning capacity upgrades 
in the future, or considering investment in a new technology or region. For 
example, Google Fiber is now expanding to Salt Lake City, from where 
we received our dataset. The analysis of change in user behavior with capacity 
estimates that a low number of users already on the 105 Mbps service tier will 
actually increase their demand beyond the 105 Mbps capacity if they 
migrate to a higher capacity service offered by Google. 

This may 
convince subscribers and policy makers alike that individually, the demand of a 
household is affected by the increase in access link capacity, especially for 
subscribers who did not have much demands. We believe that this is the 
perspective the FCC takes when considering deployment and adoption of 
broadband services. Essentially, if any change is observed in demand due to an 
upgraded service, the FCC may interpret that as \emph{adoption} of the higher 
available tier.