\section{Discussion}\label{sec:discussion}

\subsection{Different Perspectives of Utilization}

The FCC has the responsibility to increase the availability and deployment of broadband
throughout the US (with the broadband threshold benchmark defined as 25 Mbps in downlink and
3 Mbps in uplink). Their progress report states that: given the option, users will adopt a
higher tier bandwidth~\cite{}, thereby meriting the high investment. However, a survey
conducted by NCTA showed that the largest deterrent to broadband adoption is that users
do not \emph{need} broadband (the second largest is the cost). The conflicting view of 
the ISP is that the cost of deployment in an unchartered area is too high, unless
a significant number of households \emph{need} it. Thus, both parties are asking the same
question: do people \emph{need} a higher capacity, i.e., what is their \emph{utilization}
as compared to the capacity?

Previous research shows that the utilization will increase as the capacity of the access
link increases~\cite{}, and also that utilization\footnote{called demand in their work} and capacity
follow a law of diminishing returns~\cite{bischof2014broadband-behavior}. However, such
studies have been biased by studying users that actually required a higher capacity
for their usage. It is inevitable that such a correlation would exist for 
such households, whose utilization is bottlenecked by the ISP.

In this work, we ask a more fundamental question: \emph{how much does the user behavior
change with increasing capacity}. Specifically, when the capacity is already very high
and the user has not opted for an increase, does their utilization still vary with capacity?
Both the FCC and the ISPs have a different perspective of the utilization:

% the overall usage is very similar
% max peak/day are the same, but on the lower end peak usage per device changes
% much lesser than the capacity but it is still a change
\paragraph{The FCC perspective: }\emph{Utilization as adoption} of a higher capacity
link when available (but not under the constraints of a much higher cost). The answer to this
question is important to the FCC to encourage further deployment of high tier links throughout
the US. Essentially, if \emph{any} change is observed in link utilization due to the upgrade in our dataset, the FCC may interpret that as \emph{adoption} to the higher available tier.

% The max util DOES NOT change => we are not the bottleneck and users are satisfied.
\paragraph{The ISP perspective: }\emph{Utilization as a capacity bottleneck}, i.e., if
the ISP can show that the \emph{maximum utilization} of a household does not vary
with increasing capacity, it will prove there is not enough demand to offer a higher tier.  
The ISP needs the answer to this question for future capacity planning, and the cost-analysis
for the investment of new technology in any area.
For example, Google Fiber is now expanding to Salt Lake City, from where we received our dataset.
The analysis of change in user behavior with capacity will estimate the number of users that
actually \emph{need} the higher capacity service offered by Google.

Thus there is a need to measure \emph{utilization} at times when users \emph{need} the Internet
capacity the most: the peak usage.


\todo{EVERYTHING BELOW THIS IS TODO}

\subsection{User Taxonomy}


The Sandvine reports present a taxonomy of users based on their contribution to real-time entertainment traffic. We incorporate a similar definition based on contribution to data traffic, along with our observations of utilization, to present a taxonomy of the users in our dataset. One category of users is the non-utilizers, i.e., non-aggressive low bandwidth users, that contribute less than SOME THRESHOLD PERCENTILE to the daily data transferred \sg{these the ISP can ignore, also they probably don't need this tier as their utilization from the previous section must be super low}. The second category is of users contributing most aggressively to the data at the ISP \sg{these users will probably gobble up a higher capacity link if given a chance - they're the ones who effect all our graphs.. Need to check this claim}. We further subdivide this high utilizing subcategory based on differing prime-time ratio and peak-ratios follows... \todo{need to think and analyze this further: technical definition to do the analysis}
\begin{itemize}
\item Aggressive All-Time: Users having a low peak-ratio due to a lower variance. Is also expected to have a low prime-time ratio.
\item Aggressive Prime-Time: The usual streamer with a high prime-time ratio and a high peak ratio.
\item Aggressive Non-Prime-Time: Possibly a business user with a low prime-time ratio  but a high peak ratio
\end{itemize}