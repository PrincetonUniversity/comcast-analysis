\section{Empirical Analysis}\label{sec:analysis}

%The nature of the randomized controlled experiment used to collect the dataset 
%of high service tier users allows us to evaluate the effect of a 
%\emph{factor}, the service plan upgrade, as a treatment applied to a control 
%group. 

Controlled experiments are difficult to do on the Internet scale. However, in 
our work, we had access to a randomized control experiment used to collect a 
dataset on the scale of a large city. This enables us to study the effect of 
just one factor, \emph{the service plan upgrade}, while controlling the affect 
of other factors, such as price, performance, or regional differences between 
users.

By studying a particular group of high service plan users, who were 
upgraded without their knowledge, we mitigate the affect of biases that 
previous studies on usage and capacity suffer from: (a) \emph{Avoid behavioral 
change bias:} offering users with high capacity a further increase without 
their knowledge avoids the risk of behavioral changes that may occur when one 
purposefully buys a higher bandwidth connection; and (b) \emph{Avoid 
dissatisfied user bias:} we study high capacity tier users in a \control{} 
group, that are are not utilizing their access link completely, and thereby 
mitigate the effect of traffic biases that occur because subscribers' previous 
capacity was insufficient for their usage. Studying datasets with these biases 
are prone to positive correlations between usage and capacity, which we 
avoid by examining a single high capacity tier with an unannounced upgrade.

\begin{table}[ht]
\small 
\begin{tabular}{| c | c |}\hline
\textbf{Parameter} & \textbf{Definition}	\\\hline
Traffic Demand per Subs.& \(\frac{\text{total bytes transferred in 
measurement int.}}{\text{number of contributing subscribers}}\)	\\
Prime Time		& 8:00 PM - 12:00 AM   		\\
Prime Time Ratio 	& \( \frac{ \text{avg usage in peak (prime-time) 
hour}}{ \text{avg usage in off-peak hour}}\) 		\\
Peak Ratio 		& \(\frac{\text{95\%-ile of daily traffic 
demand}}{\text{mean of daily traffic demand}}\)	\\\hline
\end{tabular}
\caption{Evaluation Criteria}
\label{tab:eval-criteria}
\end{table}


\paragraph{Evaluation Criteria. } We aim to answer the question: is there a 
change in traffic demand with service tier upgrades? To define change, we use 
the evaluation criteria defined in table~\ref{tab:eval-criteria}. The traffic 
demand for a subscriber is defined as the total bytes transferred, in 
uplink or downlink, during a single measurement period (15 minutes).

Diurnal Internet usage is known to be the highest during the evening hours from 
7:00 PM to 11:00 PM \cite{fcc2015broadband-report}. The consumption of high 
bandwidth video is at its daily maximum during prime-time. Our observations 
showed that the consumption was highest for the control and treatment groups 
between 8:00 PM to 12:00 AM. To study the disparity is demand during prime-time 
and the rest of the day, we define the prime-time ratio as the ratio between 
the average demand during a prime-time hour, to the average demand outside the 
prime-time hour.

Our analysis showed that although demand increases due to the service upgrade, 
the increase during prime-time hours was insignificant. We study the daily 95 
percentile to mean ratio for subscribers to evaluate whether disparity between 
the peak (95 percentile demand) and the mean increases at times other than 
prime-time.

\subsection{Traffic Demand Per Subscriber}\label{subsec:behavior}

\begin{figure}[t]
\begin{minipage}{\linewidth}
\centering
%
\begin{subfigure}[b]{.49\linewidth}
\includegraphics[width=\linewidth]{figures/weekday_demand_mean.pdf}
               \caption{Weekday traffic demand.\label{fig:weekday-daily-usage}}
\end{subfigure}
%
\begin{subfigure}[b]{.49\linewidth}
\includegraphics[width=\linewidth]{figures/weekend_demand_mean.pdf}
               \caption{Weekend traffic demand.\label{fig:weekend-daily-usage}}
\end{subfigure}
%
\end{minipage}
\caption{Mean subscriber demand (bytes per 15-minute interval).}
\label{fig:traffic-demand-timeseries}
\end{figure}

\begin{table}[t]
\centering
\begin{tabular}{c c | c c c |}
\cline{3-5}
        &           & median & mean  & 95\%  \\ \hline
\multicolumn{1}{|c|}{\multirow{2}{*}{Weekday}} 	& treatment & 35.97  & 35.58 & 61.12 \\
\multicolumn{1}{|c|}{}					        & control   & 28.06  & 31.12 & 58.78 \\\hline
\multicolumn{1}{|c|}{\multirow{2}{*}{Weekend}}	& treatment & 45.27  & 40.10 & 64.27 \\
\multicolumn{1}{|c|}{} 					        & control   & 41.15  & 37.66 & 62.23 \\\hline
\end{tabular}
\caption{\green{Weekday and weekend traffic demands (MB) per measurement window.}}
\label{tab:traffic-demand-description}
\end{table}

We first explore how an upgrade to a higher service tier affected the
average traffic demand per subscriber, for different times of the day
and days of the week.  Figure~\ref{fig:traffic-demand-timeseries} shows
the average downlink traffic demand across subscribers for a week, for
both the treatment and control groups. We observe that subscriber
behavior differs significantly on weekdays and weekends.  The average
per subscriber demand over a weekday is 35.6 MB, and the 95th percentile
peak demand is 61.12 MB for subscribers in the treatment group
(Table~\ref{tab:traffic-demand-description}).  Over a weekend, the
average demand is 40.1 MB, and the 95th percentile demand is 64.3 MB for
treatment, but the median is 45.27 MB due to consistent use in the major
part of the day.  On weekdays, traffic demand increases monotonically
from morning until prime-time hours in the evening. On weekends, we
observed a sharp rise in demand in the early morning period, from 8:00
a.m. to 10:00 a.m. Then, the demand plateaued until the next rise
before evening prime-time hours. Previous reports indicate that
the aggregate traffic volume for US fixed access link providers usually
troughs during mid-afternoon hours (between
2:00~p.m.--6:00~p.m.)~\cite{sandvine20141h}. In contrast to these
  previous reports, we do not observe such troughs in 
subscriber demand.

\begin{figure}[t]
\begin{minipage}{\linewidth}
\centering
\begin{subfigure}[b]{.49\linewidth}
\includegraphics[width=\linewidth]{figures/cdf_peak_demand-overall.pdf}
               \caption{\label{fig:CDF-data-rate-perc95}}
\end{subfigure}
\begin{subfigure}[b]{.49\linewidth}
% \includegraphics[width=\linewidth]{figures/diff_perc95_bytes_subsc-overall.pdf}	% 5 percentile
\includegraphics[width=\linewidth]{figures/diff_perc95_bytes_subsc-overall_01.pdf}		% 1 percentile
               \caption{\label{fig:diff-peak-overall}}
\end{subfigure}
%
\end{minipage}
\caption{95th percentile traffic demand (bytes per 15 minutes) per
  subscriber for the control and treatment groups over the three-month measurement period: (\protect\subref{fig:CDF-data-rate-perc95}) Peak (95\%) traffic demand per subscriber; (\protect\subref{fig:diff-peak-overall}) Change in overall peak (95\%) demand per subscriber. Subscribers were considered at every 5\% in each group. y-axis units are bytes transferred in the peak 15-minute interval, in MB. \label{fig:traffic-demand-overall}}
\end{figure}

% Figure \ref{CDF-data-rate} shows the data rate for each measurement period and every subscriber.
Figure \ref{fig:CDF-data-rate-perc95} shows the distribution of the the
95th percentile downlink traffic demand over the three-month measurement
period. The highest peak demand per 15-minute interval amongst
subscribers in the control group was 2.97 GB; in the treatment group,
the highest peak demand was 3.0 GB.
The average peak traffic demand was 169.8 MB for control and
186.6 MB for treatment. Given the 105 Mbps service-tier capacity,
this means that users rarely utilize their links,
even on averaging the 95th percentile demand (average utilization was 
1.43\% for control and 1.5\% for treatment).

We suspected that the subscribers who downloaded most bytes in the
higher service tier would be the ones causing the largest difference in
mean demand, as previous studies have observed such a phenomenon. In
fact, we observed that the more moderate (median) subscribers actually seemed
to exhibit larger differences in traffic demand: The median peak demand
was 66.7 MB for the lower service tier, and 98.4 MB for the higher
tier.  \red{This result indicates that the more moderate subscribers who
received a service-tier upgrade significantly altered their peak
demand.}\sgfoot{Not sure if this needs change or not}
We also observed a significant difference in the mean peak
demand was present in the 50\% of subscribers in the control group
with the lowest traffic demand when compared to the same set of
subscribers of the treatment group. (This disparity appears as a large
gap under the 50\% tick in Figure~\ref{fig:CDF-data-rate-perc95}.)


\green{Figure~\ref{fig:diff-peak-overall} shows another way of looking at this
phenomenon: it explores the difference between the distribution of users
with particular traffic demands in the control and treatment groups.}
For each group, we sort the subscribers according to
increasing demand.  Then we compute the difference in peak demand for
each percentile in the group.  For example, the plot shows the median
user (50\% on the x-axis) increased their peak demand by about 25\% in
response to the service tier upgrade.  Comparing the 70\% subscribers of
both groups with the least demand, we see that peak demand in the
treatment group is higher than the peak demand in the control group,
\red{indicating that in fact even moderate users increase their demand as a
result of the service-tier increase, even though they are not using the
full capacity in either case.}
\sgfoot{Not sure if this needs change. It makes sense as a take-away message.}
 When we combine this analysis with that in
Figure~\ref{fig:CDF-data-rate-perc95}, we find that these subscribers
who respond with increased usage have a peak demand less than 200~MB.
\green{Naturally, the small number of users with the highest demand (closer to
100\%) also show a substantially larger usage for the higher service-tier.}


%\if 0
%\begin{figure}[t]
%\centering
%\includegraphics[width=.5\linewidth]{figures/cdf_diff_perc95_bytes_subsc-overall.pdf}
%               \caption{Distribution of the difference between \treatment{} and \control{}
%               peak traffic demands\label{fig:cdf-diff-perc95}}
%\end{figure}
%
%Figure~\ref{fig:cdf-diff-perc95} shows the distribution of the
%difference in demands of the two groups by percentile. There are 15\%
%subscribers in the \treatment{} that have peak demands have peak demands
%that are similar, or lower than the equivalent 15\% of subscribers in
%\control{}. Figure~\ref{fig:diff-peak-overall} shows that this group
%lies around between the 65 and 80 percentile
%demand. Figure~\ref{fig:CDF-data-rate-perc95} shows that between the
%65-80 percentile mark, subscribers had a demand between 200~MB to
%400~MB. For these subscribers in the treatment group, the equivalent
%subscribers in the control group had demands up to 10~MB higher. 20\% of the
%differences between the peak of the treatment group and the control
%group are higher than 30 MB. These subscribers lie between the between
%30\% and 50\% difference. Their peak demands are between 40--100
%MB as part of the higher tier, and 10--70~MB if they belong to the lower
%tier.  The subscribers with the highest traffic demand in the treatment
%group (beyond 800 MB peak demand) have demands 15--30~MB more than the
%highest peak demanding subscribers in the control group.  In the case of
%upstream traffic, 90\% of users in the treatment group consistently
%upload 2 MB more traffic than the control group. The top 10\% of the
%highest demanding subscribers in the treatment group uploaded between 2--10~MB
%more than the top 10\% of the control group.
%\fi

\begin{figure}[t]
\begin{minipage}{\linewidth}
\centering
\begin{subfigure}[b]{.49\linewidth}
 \includegraphics[width=\linewidth]{figures/cdf_peak_demand-daily.pdf}
                \caption{\label{fig:CDF-data-rate-daily-perc95}}
 \end{subfigure}
\begin{subfigure}[b]{.49\linewidth}
\includegraphics[width=\linewidth]{figures/diff_perc95_bytes_subsc-daily-overall_01.pdf}		%1 percentile
%\includegraphics[width=\linewidth]{figures/diff_perc95_bytes_subsc-daily-overall.pdf}		%5 percentile
                \caption{\label{fig:diff-peak-daily}} 
\end{subfigure}
%
\end{minipage}
  \caption{\green{95th percentile traffic demand (bytes per 15 minutes) per day
  per subscriber for the control and treatment groups: (\protect\subref{fig:CDF-data-rate-daily-perc95}) Daily peak (95\%) traffic demand per subscriber; (\protect\subref{fig:diff-peak-daily}) Change in daily peak (95\%) demand per subscriber. Subscribers were considered at every 5\% in each group. y-axis units are bytes transferred in the peak 15-minute interval, in MB.\label{fig:traffic-demand-daily}}}
\end{figure}


\red{Further investigation revealed that users with moderate peak traffic
demands not only exhibit a large difference in their traffic demands in aggregate, but also
on a daily basis.} \origfoot{Further investigation revealed that users with moderate peak traffic
demands not only increase their traffic demands in aggregate, but also
on a daily basis.}  Figure~\ref{fig:traffic-demand-daily} shows that
when subscribers on the lower tier had a daily peak demand under 600~MB,
70\% of subscribers in the treatment group had 15-minute demands that
were 5--20~MB higher.  The ratio of the differences in demand across
percentiles also shows that the \red{40\% of subscribers with lowest peak
demands in the treatment group demonstrate more than
double the daily peak traffic demand of the control group.}
\origfoot{40\% of subscribers with lowest peak
demands in the control group more than
double their daily peak traffic demand in response to service-tier upgrades}

\red{One possible explanation for why moderate users increase their usage
in response to a service-tier upgrade is that the higher service tier
not only affords more capacity, but also a better user experience
(\eg, faster downloads).}\sgfoot{Don't this needs change in current context.}
\red{Thus, even though users may not be
exhausting the capacity of the higher service tier, they nonetheless
seem to respond to the service tier upgrade by using the Internet more
than they had before the service-tier upgrade.} \sgfoot{How to fix this one?}

%
%\if 0
%There could be many reasons for this increase in demand for subscribers with
%lower peak demands. One reason could be short term web activities (such as short videos,
%web browsing) that have a slightly higher traffic demand. Such an increase in demand would
%then be apparent during hours when users pursue such activities (mostly prime-time).
%Another reason could be an increase in background traffic (such as software updates).
%Studying the applications responsible for such behavioral changes in traffic demand is not
%in the scope of our current work. However, an interesting question that arises is: when
%does the demand increase the most throughout a day? Our analysis showed that the most
%subscribers reach within 95\% of daily maximum demand between 8:00 p.m.--12:00 a.m.
%\fi

% \begin{table}[h]
% \begin{tabular}{lllll}
%         &           & 1     & 2     & 3     \\
% Weekday & treatment & 21:45 & 00:00 & 23:30 \\
%         & control   & 22:30 & 22:15 & 00:00 \\
% Weekend & treatment & 23:30 & 23:45 & 14:30 \\
%         & control   & 22:30 & 21:30 &      
% \end{tabular}
% \end{table}


\subsection{Peak Traffic Demand}\label{subsec:peakratio}

\begin{figure}[ht!]
%\hspace*{-0.2in}
\begin{minipage}{0.90\linewidth}
\centering
\begin{subfigure}[b]{0.90\linewidth}
\includegraphics[width=\linewidth]{figures/cdf-max-per-device.png}
  \caption{CDF of max per device: test set has higher (max) average data rate 
below 10 kbps.  30\% of devices in the control set have a max data rate of 2 
kbps while 30\% of test set has a max data rate of 10 kbps. (sanity check 
numbers, redo plot)}
  
%http://sites.noise.gatech.edu/~sarthak/files/comcast/plots/full_dw/cdf-max-per-
%device.png
\label{fig:CDF-data-rate-max}
\end{subfigure}
%
\vspace{-1em}
%
\begin{subfigure}[b]{0.90\linewidth}
\includegraphics[width=\linewidth]{figures/cdf-max-per-day-per-device.png}
  \caption{CDF of max per device daily}
  \vspace{1em}
  
%http://sites.noise.gatech.edu/~sarthak/files/comcast/plots/full_dw/cdf-max-per-
%day-per-device.png
  \label{fig:CDF-data-rate-max-daily}
\end{subfigure}
%\hfill
\end{minipage}
\caption{Peak Utilization: The maximum data rate varies for test and control 
set 
for low data rates, and this variation is present daily.}
\label{fig:peak-utilization}
% created using docs/metadata-separated.log
\end{figure}



Based on the measurement methodology, we study the highest utilization seen by a 
household
both in its lifetime, and on each day. Our aim is to examine the peak usage per 
household, and
study if the behavior changes due to an upgrade.

Figure~\ref{fig:CDF-data-rate-max} provides a distribution of the highest 
average
data rate a household achieves. To avoid outliers, we also plot the 90\%-ile of 
the max
data rate achieved by households in both \test and \control sets. We see that a 
median
household is expected to achieve the highest data rate of between 1 -- 10 Mbps 
over its
lifetime. This is much lower than the access link capacity,
indicating that the median device has a utilization ratio (avg data 
rate:capacity) under
0.1 in our dataset. The number of households that increased their peak 
utilization beyond
the \control set's 105 Mbps capacity were negligible.

Surprisingly, we see that 30\% of the households from the \test set have a low 
peak
utilization (under 0.1 Mbps), while 40\% of the \control set households are 
under 0.1 Mbps.
Thus, the absolute peak utilization does not increase when compared to the 
access link
capacity, but there is certainly an increase in peak utilization of devices that 
had a low
requirement, due to the change in capacity.

To investigate this further, we also study the \emph{peak utilization per device 
on a daily basis}.
Figure~\ref{fig:CDF-data-rate-max-daily} shows that for 30\% of the devices,
the maximum data rate in the \test set is consistently higher than the \control 
set, albeit
no where near the actual access link capacity.

This is similar to the behavior observed in figure~\ref{fig:TS-data-rate-daily}, 
showing that the peak
usage during prime-time is unaffected, but lower utilization throughout the day 
is higher
for the test set. We speculate that there could be two possible reasons for this 
increase
in utilization: (1) short term downloads and/or web browsing achieves a slightly 
better
data rate on a small time scale, or (2) real-time video quality is slightly 
higher, but
not enough to completely saturate the access link capacity. Unfortunately, we 
miss these
short lived, or consistent, events due to a 15 minute time slot granularity and 
only
looking at byte counters.



\paragraph{Peak Ratio: }The Sandvine Reports show that although the mean usage has remained
stable for the past few years, usage during peak-times has increased
drastically~\cite{sandvine20141h}. To measure this growth, they introduce the
concept of peak period, measured when the network is within 95\% of its highest point.
Although, these reports present a good view into aggregate usage patterns over a month,
they neglect to analyze usage characteristics individually. Inspired by their
definition, we measure the disparity between the 90 percentile of the peak and median
usage of each household within a day, and call this the \emph{Peak-Ratio}. In
section~\ref{subsec:peakratio} we show that the peak ratio can be used
to divide users in the same tier based on their usage behavior.

% the ratio of the 90\%-ile to the median throughput per day.


\begin{figure}[ht]
\begin{minipage}{0.90\linewidth}
\centering
\includegraphics[width=1\linewidth]{figures/peakratio-CDF-devices-MEDIAN.png}
\caption{Median peak ratio per device showing that test set has higher daily ratio (50 times by median). Thus ISPs should condition their networks to 50 times the median usage for each user added in the worst case scenario.}
%http://sites.noise.gatech.edu/~sarthak/files/comcast/plots/full_dw/peakratio-CDF-devices-MEDIAN.png
\label{fig:CDF-peak-ratio-median}
\end{minipage}
\end{figure}

To further characterize and compare the deviation of data rate for the \control and \test set, we examine \emph{peak-ratio} as defined above. 
Figure ~\ref{fig:CDF-peak-ratio-median} shows that the median peak-ratio for each device in the \test set is much larger than that of the \control set.
\todo{replace much larger with the exact number or percentage}.
\sg{Taken together} with our observations of a lower prime-time ratio of the \test set (section~\ref{subsec:primetime}) this implies that there are households in the \test set that achieve a peak-ratio $>$ 1, but not during the prime-time hour. We believe that these households might actually be small businesses or work-at-home users that peak during daytime hours instead of evening hours.

\begin{figure}[ht]
\begin{minipage}{0.9\linewidth}
\centering
\includegraphics[width=0.9\linewidth]{figures/cdf-prime-time-ratio[replace].png}
\caption{(old) Prime Time ratio + usage can be used to divide users into four sets: aggressive all time + non aggressive all time, aggressive peak time, aggressive non-peak time (business hours). 30\% PT $<$ 1: possibly businesses with normal work-hours . 20\% PT $>$ 2: aggressive prime-time streamers}
%http://riverside.noise.gatech.edu:8083/separated/full/cdf-prime-time-ratio-per-device.png\\
%http://riverside.noise.gatech.edu:8083/plots/full_dw/prime-time-ratio-per-device-cdf-ALL.png
\label{fig:CDF-prime-time-ratio}
\end{minipage}
\end{figure}

The median peak-ratio per device itself shows a large range, from 1 to 10e6 (figure~\ref{fig:CDF-peak-ratio-median}), and the maximum peak-ratio per device was an order higher. Clearly there are some households that have a very even usage throughout the day (low peak ratio), and others that are extremely aggressive only at certain times (high peak ratio). We plot this segregation in figure ~\ref{fig:CDF-prime-time-ratio}.


% other results:
%big difference (2 x median ratio) in per device per day ratios of 90%ile:median.
%weird shape again for values < ratio 100
%big difference in this ratio per day, and it is consistent across all individual sets + months.
%very large for Dec, slightly smaller for Nov
%interestingly, at higher ratios control is slightly > test. This means that certain devices in control set have a huge std (ratio) in a day as compared to test set which has a lower “max” ratio.

\todo{ TO PLOT? :}
\begin{itemize}
\itemsep0em
\item peak ratio cdf vs no of devices
\item peak ratio cdf vs time of day where peak occurred
\item no of devices cdf vs time of day where peak occurred
\end{itemize}

Based on differing usage profiles within the same high tier bandwidth, we suggest that the FCC adopt multiple benchmarks based on usage characteristics to better characterize broadband availability, deployment, and adoption in the US. Such multiple benchmarks can be the minimum broadband speed required per user based on the kind of traffic expected during a day. ISPs can also offer these users better plans based on hour-of-the-day or usage caps to encourage more off-peak usage. These users probably don't cause latency spikes in PT.

\subsection{Prime Time Ratio}
\label{subsec:primetime}

In section ~\ref{sec:methodology}, we established the importance of measuring behavior during prime time hours. To reiterate, the prime time hours as defined by the FCC is the period between 7:00 PM -- 11:00 PM on weeknights when the network is under peak load~\cite{fcc2014measuring-broadband}. \todo{The first occurrence of this definition was ... should this be in methodology?? }

Our analysis of the peak usage in figure~\ref{fig:TS-data-rate-daily} motivated us to revisit this definition. We observed that in our dataset the four hour period 8:00 PM -- 12:00 AM is a much better indicator of prime-time hours than the FCC's original definition. This discrepancy could be limited to high tier households only, or it could imply that due to the flexibility of video-on-demand, consumers across the country are opting for a later time to watch real-time-entertainment. 

\begin{figure}[ht!]
\begin{minipage}{\linewidth}
\centering
\includegraphics[width=\linewidth]{figures/prime-time-ratio-by-date[replace].png}
\caption{Prime Time ratio showing weekly pattern + differences during holiday periods (Thanksgiving, Christmas)}
%http://riverside.noise.gatech.edu:8083/separated/full/prime-time-ratio-by-date.png
\label{fig:TS-prime-time-ratio}
\end{minipage}
\end{figure}

We use the updated definition of prime-time (as 8:00 PM -- 12:00 AM) to examine the aggregate prime-time ratio each day for both \test and \control sets. Figure ~\ref{fig:TS-prime-time-ratio} shows that the prime-time ratio for the \test set is 10\% less than the control set, implying that the ratio of average data rates in peak (prime-time) hours and off-peak hours in a day reduces due to the capacity upgrade from 105 Mbps to 250 Mbps. The prime-time ratio can decrease only in two scenarios (1) the usage during prime time decreases, and off-peak remains the same, or (2) the usage during off-peak hours increased due to the upgrade.

%measures the variance in prime-time hours and off-peak hours
We also noticed that multiple agencies define prime-time differently causing difficulty in comparing studies. We recommend that the FCC studies prime-time usage behavior and standardizes its definition. If supported by analysis, the FCC should update prime-time to a later hour.%Another definition might be local to each agent examining their peak network load as the approx time when aggregated traffic is 90\%-ile of the maximum load, which can cause latency delays. \todo{ should this be in the discussion instead }


%\subsection{Aggressive Subscribers}\label{subsec:prevalence}

\begin{figure}[ht]
\begin{minipage}{\linewidth}
\centering
\includegraphics[width=\linewidth]{figures/prevalence.png}
\caption{User prevalence as threshold increases.}
\label{fig:prevalence}
\end{minipage}
\end{figure}

Outliers. Prevalence.
While the general behavior did not change, still 11 users maxed out traffic.
For example (explain the fig)