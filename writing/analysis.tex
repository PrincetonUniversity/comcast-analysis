\section{Empirical Analysis}\label{sec:analysis}

\subsection{Importance of Measuring Peak Usage}

Internet usage throughout a day follows diurnal sleep-patterns, and researchers
have shown that such patterns are in fact correlated with GDP, Internet allocations,
as well as electrical consumption of a region~\cite{ant-diurnal-web}. This makes
the study of usage behavior extremely relevant to the governmental bodies responsible
for development, such as the FCC, when considering policy decisions. 

\paragraph{Prime Time: }The daily diurnal nature of usage patterns across many
households naturally requires the provider to design networks capable of handling 
load at the peak times in a day. Such peak times are usually observed during
evening hours, and the data transferred at this time is called peak usage.
The FCC defines \emph{Prime Time} as the local time from 7:00 PM to 11:00
PM~\cite{fcc2014measuring-broadband}, when many
households heavily consume real-time entertainment traffic (video), seen as primarily
responsible for high usage during these hours. Latency and performance are adversely
affected during prime-time, causing bottlenecks at home, the last mile, in
transit, or at the content server. For example, the Sandvine Global
Internet Phenomena Report \footnote{The Sandvine Reports ~\cite{sandvine20141h,
sandvine20142h}are released bi-annually and
contain a detailed analysis of aggregate Internet usage. They are also referred
to in the FCC reports~\cite{fcc2015progress-report, fcc2014measuring-broadband,
fcc2014progress-report}} showed that devices in the same household selected Netflix's
own CDN (OpenConnect) during off-peak hours, and third party CDNs (with differing
performance) during prime-time. This may happen because Netflix OpenConnect is over-utilized
during prime time~\cite{sandvine20141h}.

\paragraph{Prime Time Ratio: }To measure the concentration of network usage during prime time,
Sandvine defined the \emph{Prime-Time ratio} as the ``absolute levels of network traffic
during an average peak period hour with an average off-peak hour''. Based on the FCC
definition of prime-time hours (7p-11p), we measure the daily prime-time ratio of $set_{full}$
in section~\ref{subsec:primetime}.

\paragraph{Peak Ratio: }The Sandvine Reports show that although the mean usage has remained
stable for the past few years, usage during peak-times has increased
drastically~\cite{sandvine20141h}. To measure this growth, they introduce the
concept of peak period, measured when the network is within 95\% of its highest point.
Although, these reports present a good view into aggregate usage patterns over a month,
they neglect to analyze usage characteristics individually. Inspired by their
definition, we measure the disparity between the 90 percentile of the peak and median
usage of each household within a day, and call this the \emph{Peak-Ratio}. In
section~\ref{subsec:peakratio} we show that the peak ratio can be used
to divide users in the same tier based on their usage behavior.

In the next section, we analyze spacio-temporal network usage behavior:
\textbf{Time Series Behavior (TS): }aggregating the usage (and utilization) per household
over time (daily or weekly, per time slot). \textbf{Distribution Across Devices (CDF): }aggregating over time slots per day to measure utilization per device. We use the prime-time ratio and peak usage
as criteria to evaluate usage behavior and interpret utilization.


\section{Results}
\label{sec:results}
Q/A\\
Plots

