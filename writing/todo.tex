data

- relevance of data and why observations here can be applied to higher tier residences in other places


results

- put a note on why studying asymmetry of these homes is important heavy users


discussion


intro

In essence offering more cap will increase usage, making the selection of speed benchmarks tough. fcc wants to increase the bb as users will utilize the extra capacity if offered. isps get pissed as they do not qualify as bb and do not see a future in which users will adopt the higher broadband offered. In fact, a survey by fcc says need is first factor and price is second in low adoption of bb.

This is useful for FCC which holds responsibility to ensure advanced broadband deployment throroughout the US, and takes actions to accelerate deployment by removing barriers to investment and promoting competition. The most important - will people adopt broadband if offered - and our study goes into this question - do users keep adopting to a higher tier when offered, if price and capacity do not increase, or is there a limit to the demand because of saturation in service or bottlenecks elsewhere.

Recent growth in Internet services and adoption throughout the US promted the Federal Communications Commission (FCC) to increase the broadband speed benchmark to 25/3 Mbps\footnote{explain format}. Furthermore, the FCC is now considering usage demand as one of the benchmarks they would monitor when evaluating broadband deployment \cite{fcc2015broadband-report}.