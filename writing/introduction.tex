\section{Introduction}\label{sec:introduction}

importance of usage. important for fcc. important for isps capacity planning. bottlenecks to usage can be the user themselves, the transit, or the service.

previous study shows usage is tough to determine. choices are random too many variables [36]. dasu guys study usage with many factors by monitoring transfer bytes throughout the world. designed as a natural exp they show that apart from capacity, price and performance (latency pack loss) also play an imp role in determining adoption. the law of diminishing returns when accounting for other factors.

In essence offering more cap will increase usage, making the selection of speed benchmarks tough. fcc wants to increase the bb as users will utilize the extra capacity if offered. isps get pissed as they do not qualify as bb and do not see a future in which users will adopt the higher broadband offered. In fact, a survey by fcc says need is first factor and price is second in low adoption of bb.

we study the change in usage of subscribers who do not need any more bb are offered more without their knowledge by controlled exp. these 2219 dev are selected randomly from the same locations and same tier and ISP (keeping price and performance latency and ploss similar). Compared with 18355 devices sharign the same demographics but getting what they paid for, hopefully their demands and choices are actually similar due to such control. Allows an ISP to plan for capacity based on trends in usage demand - if demand don't increase users are satisfied with high tier.

This is useful for FCC which holds responsibility to ensure advanced broadband deployment throroughout the US, and takes actions to accelerate deployment by removing barriers to investment and promoting competition. The most important - will people adopt broadband if offered - and our study goes into this question - do users keep adopting to a higher tier when offered, if price and capacity do not increase, or is there a limit to the demand because of saturation in service or bottlenecks elsewhere.

Our contributions show the complicated nature of individual usage and capacity, and motivate the need of demand based tiers and benchmarks in the future. We see off peak usage increases for users not utilizing the link anyway. We also seee that in our high speed tier dataset, prime time hours become late (8-12). Also peak usage increases for low util users daily. In the dataset over three months only 11 users out of 1500 actually went over 105 Mbps.

The analysis of usage behavior to discuss different perspectives of bb utilization. We see that greedy rich users do not increase their usage so isps may conclude that they are satisfying their customers fully. But we also see that frugal rich users increase usage, especially in off peak, so conclude that as a consumer or policy maker capacity is still impacting perf. Both parties may have opposing views but look at the same observation. Motivates further study of the relationship between bb and capacity, esp. with similar control exp, and find out why this increase happened? Also look into usage demands as a metric used to determining broadband benchmarks.

Recent growth in Internet services and adoption throughout the US promted the Federal Communications Commission (FCC) to increase the broadband speed benchmark to 25/3 Mbps\footnote{explain format}. Furthermore, the FCC is now considering usage demand as one of the benchmarks they would monitor when evaluating broadband deployment \cite{fcc2015broadband-report}.