\section{Introduction}\label{sec:introduction}

The large impact of broadband Internet in our daily lives, and a rapid increase 
in offered services has enabled have motivated interested parties, such 
researchers, policy  makers and ISPs, to the complex relationship between 
traffic demand and broadband capacity. For users on a lower tier, upgrades in a 
service plan result in an increase in traffic demand. However, with the growing 
development of bandwidth intensive applications, increasing video content, 
coupled with a recent boom in Internet enabled consumer goods, it is important 
to study how traffic demand will increase for consumers who are currently on 
high tier service plans and contribute the most to Internet traffic.
% + more work @ home
% + fcc wants to add usage as a bb parameter

Previous work has shown that users who are already maximizing their usage on a 
given access link will continue to do so when they are migrated to a higher 
service tier. The focus of our work is to study the traffic demand 
of subscribers who are already on service plans with high downstream 
throughput capacity, and their response to an undisclosed service plan upgrade 
as part of a randomized controlled trial (RCT). Our dataset offers the 
unique opportunity to explore the impact on traffic demand while mitigating 
the cognitive bias of the service tier upgrade by withholding that information 
from the subscriber. To the best of our knowledge, this is the first such 
comparative study of usage behavior in a controlled experiment to study usage 
behaviors.

Our study is based on Comcast's analytic data collected from home gateways for 
their customers in Salt Lake City, Utah. This data consists of byte transfers
collected continuously every 15 minutes from two types of users; control: users 
that pay and use a high capacity access link (105 Mbps), and treatment: users 
that pay for 105 Mbps but are actually offered a much higher capacity access 
link (250 Mbps) without their knowledge. This decision was specifically 
anticipated to enable researchers to investigate the question: does increase in 
capacity change traffic demand? Our analysis shows that not only are we capable 
of answering the above question, but we can also \todo{validate and improve 
more FCC policies on broadband benchmark.}

We evaluate a total of three months of traffic demand for more that 20000 
individual subscribers, 2200 of whom had access to a 2.5-fold increased 
capacity tier. Our results show that while the aggregate and mean daily usage 
did not change significantly, there is an observable increase in the 95 
percentile traffic demand for the higher capacity users. However, this 
increase occurs mostly in the off-peak hours, when subscribers are not expected 
to utilize the link. \todo{confirm this by taking the ratio of avg or median 
usage per time-slot in treatment and control}
Furthermore, we observed that subscribers who are already using most of their 
available capacity do not use significantly more capacity when they are moved 
to a higher tier. In contrast, subscribers with a low traffic demand increase 
their daily 95 percentile peak usage by 10\%. Finally, we observed that 
more than than 99\% of the users do not even use the increased service tier 
consistently for 15 minutes at any time in the three months of data we study.
% + prime time hours were 8-12 instead of 7-11 for high tier dataset
% + asymmetry?

With the recent interest of the Federal Communications Commission (FCC) in 
adopting usage (traffic demand) as one of the broadband benchmarks, we discuss 
the possible implications of our observations for both policy makers, who 
determine broadband deployment policies for the US based on consumer demands, as 
well as ISPs such as Comcast, who plan broadband deployment and capacity 
upgrades based on trends in aggregate usage behaviors. We conclude that both 
policy makers and ISPs may have different perspectives of the same observation, 
and motivate the need to further study traffic demand as a broadband benchmark 
metric.


The rest of the paper is organized as follows. In \autoref{sec:related} we 
list some previous studies of traffic demand. Then, in \autoref{sec:data}, 
we offer details about our data, sanitization, and characterization. We then 
proceed by describing our evaluation criteria and examine changes in broadband 
usage [\autoref{sec:analysis}]. \todo{We do this by studying spacio-temporal 
usage patterns in terms of peak utilization, prime-time ratio, asymmetry, and 
prevalence.} Lastly, we comment on the differing perspectives of traffic demand 
in \autoref{sec:discussion}, and summarize our findings in 
\autoref{sec:conclusion}.
% + spacio temporal analysis in results
% + user taxonomy in discussion?