\section{Introduction}
\label{sec:intro}
\begin{itemize}
\item FCC hypothesis
\item Answer from an ISP perspective
\item Current usage
\end{itemize}

%Paragraph 1: Motivation. At a high level, what is the problem area you are working in and why is it important? It is important to set the larger context here. Why is the problem of interest and importance to the larger community?

%Paragraph 2: What is the specific problem considered in this paper? This paragraph narrows down the topic area of the paper. In the first paragraph you have established general context and importance. Here you establish specific context and background.

%Paragraph 3: "In this paper, we show that ...". This is the key paragraph in the intro - you summarize, in one paragraph, what are the main contributions of your paper given the context you have established in paragraphs 1 and 2. What is the general approach taken? Why are the specific results significant? This paragraph must be really really good. If you can't "sell" your work at a high level in a paragraph in the intro, then you are in trouble. As a reader or reviewer, this is the paragraph that I always look for, and read very carefully.

%You should think about how to structure this one or two paragraph summary of what your paper is all about. If there are two or three main results, then you might consider itemizing them with bullets or in test (e.g., "First, ..."). If the results fall broadly into two categories, you can bring out that distinction here. For example, "Our results are both theoretical and applied in nature. (two sentences follow, one each on theory and application)"

%Paragraph 4: At a high level what are the differences in what you are doing, and what others have done? Keep this at a high level, you can refer to a future section where specific details and differences will be given. But it is important for the reader to know at a high level, what is new about this work compared to other work in the area.

%Paragraph 5: "The remainder of this paper is structured as follows..." Give the reader a roadmap for the rest of the paper. Avoid redundant phrasing, "In Section 2, In section 3, ... In Section 4, ... " etc. 