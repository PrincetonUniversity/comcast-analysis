\section{Introduction}\label{sec:introduction}

With the large impact of broadband Internet on our daily lives and 
its rapid increase in bandwidth-intensive services, policymakers and 
service providers (ISPs) are trying to determine how much bandwidth 
consumers need. With the proliferation of high 
quality video content, and the recent boom in Internet-enabled consumer 
device, subscribers who are currently satisfied with higher service plans 
may not continue to be so. 
% + more work @ home
% + fcc wants to add usage as a bb parameter

Previous work studying the complex relationship between demand and capacity has 
shown that users who are already maximizing their usage on a given access link 
will continue to do so when they are migrated to a higher service tier. The 
focus of our work is to study the traffic demand of subscribers who are already 
on service plans with high downstream throughput capacity, and their response to 
an undisclosed service plan upgrade as part of a randomized controlled trial 
(RCT). Our dataset offers the unique opportunity to explore the impact on 
traffic demand while mitigating the cognitive bias of the service tier upgrade 
by withholding that information from subscribers. To the best of our knowledge, 
this is the first such comparative study of usage behavior in a controlled 
experiment to study responses to service upgrades.

Our study is based on Comcast's analytic data collected from home gateways for 
their residential broadband customers in Salt Lake City, Utah. This data 
consists of byte transfers collected continuously every 15 minutes from two 
types of users; control: users that pay and use a high capacity access link (105 
Mbps), and treatment: users that pay for 105 Mbps but are actually offered a 
much higher capacity access link (250 Mbps) without their knowledge. This 
decision was specifically anticipated to enable researchers to investigate the 
question: \emph{does increase in capacity change traffic demand}?
%Our analysis shows that not only are we capable of answering the above 
%question, but we can also 

We evaluate three months of traffic demand for more than 20,000 
individual subscribers, 2,200 of whom had access to a 2.5-fold increased 
service capacity. Our analysis shows that while the aggregate and mean daily 
usage did not change significantly due to the upgrade, we observe a daily 20\% 
increase during the off-peak hours, when subscribers are not expected to fully 
utilize the link. Subscribers who are already using most of their 
available capacity in 105 Mbps, do not use significantly more capacity when
they are moved to a higher tier. In contrast, subscribers with a low traffic demand 
increase their daily 95 percentile peak usage by 10\%. Finally, we observed 
that more than than 99\% of the users do not even use the increased service 
tier consistently for 15 minutes at any time in the three months of data we 
study.
% + prime time hours were 8-12 instead of 7-11 for high tier dataset
% + asymmetry?

Following the Federal Communications Commission (FCC) recent interest in 
adopting usage (traffic demand) as one of the broadband benchmarks, we discuss 
the possible implications of our observations on traffic demand for both policy 
makers, who determine broadband deployment policies for the US based on consumer 
demands, as well as ISPs such as Comcast, who plan broadband deployment and 
capacity upgrades based on trends in aggregate usage behaviors. We conclude that 
both policy makers and ISPs may have different perspectives of the same 
observation, and motivate the need to further study traffic demand as a 
broadband benchmark metric.

The rest of the paper is organized as follows. In \autoref{sec:related} we 
overview some previous studies of traffic demand and service capacity. Then, in 
\autoref{sec:data}, we offer details about our data, sanitization, and 
characterization. We then proceed by describing our evaluation criteria and 
analyze traffic demand in response to a service tier upgrade in 
\autoref{sec:analysis}.
Lastly, we comment on the differing perspectives of traffic demand in 
\autoref{sec:discussion}, and summarize our findings in 
\autoref{sec:conclusion}.
%\sgfoot{We do this by studying spacio-temporal usage patterns in terms of peak 
%utilization, prime-time ratio, asymmetry, and prevalence.}
% + user taxonomy in discussion?
