\section{Introduction}\label{sec:introduction}

With the large impact of broadband Internet on our daily lives and its
rapid increase in bandwidth-intensive services, policymakers and service
providers (ISPs) are trying to determine how much bandwidth consumers
need. With the proliferation of high quality video content, and the
recent boom in Internet-enabled consumer device, it is worth
studying---and continually re-evaluating---whether (and how) users
consume the capacity that ISPs offer.  Up to a certain point, users will
exhaust available capacity, and they will also adapt when more capacity
becomes available; this increased demand in turn drives provisioning.
Above certain speeds, however, the typical user no longer exhausts the
available capacity. At what speed does this inflection point occur?  How
do users adapt their demands when an ISP offers faster speed tiers?
Answers to these questions will ultimately help inform policymakers and
ISPs determine how to make investments in infrastructure, and when to
make them.

In thie United States, the Federal Communications Commission (FCC) is
interested in the relationship between demand and capacity for several
reasons.  First, the FCC recognizes the need to define broadband
benchmarks based on traffic demand and is considering doing
so~\cite{fcc2015broadband-report}. It has defined a ``typical''
household traffic demand to enable concurrent broadband use, such as
video streaming, web browsing, and VoIP. Currently, the FCC is asking
for comments and suggestions on how to define such a demand-based
benchmark for future planning~\cite{fcc2015progress-report,
  fcc2014progress-report}.  Second, recent research shows that diurnal
Internet usage patterns are correlated with GDP, Internet allocations,
as well as electrical consumption of a
region~\cite{ant-diurnal-web}. This makes the study of usage extremely
relevant to the regulatory bodies responsible for development.  Finally,
the FCC is is responsible for increasing broadband deployment throughout
the US, and it recently decided to aggressively increase the broadband
threshold benchmark to 25 Mbps in downlink and 3 Mbps in uplink.  Yet, a
survey conducted by NCTA (for the FCC) showed that the largest deterrent
to deployment of faster speed tiers is that consumers do not \emph{want}
the faster speeds (the second largest deterrent is the
price)~\cite{fcc2015progress-report}. Clearly, this question deserves
both rigorous and continuous study.

% + more work @ home
% + fcc wants to add usage as a bb parameter

Previous work discovered that users who are already maximizing their
usage on a given access link will continue to do so when they are
migrated to a higher service tier~\cite{dasu-imc2014}. In this paper, we
study how the traffic demands of subscribers who are {\em already} on service
plans with high downstream throughput respond to an undisclosed service
plan upgrade as part of a randomized controlled trial (RCT). This
experiment offers the unique opportunity to explore the effects of a
service-tier upgrade on user traffic demand while mitigating the
cognitive bias of the service-tier upgrade by withholding that
information from subscribers. To the best of our knowledge, this is the
first such comparative study of usage behavior in a controlled
experiment to study responses to service upgrades.

Our study is based on data collected from the residential home gateways
of Comcast subscribers in Salt Lake City, Utah. To measure traffic
demand, Comcast collects aggregate byte counts every 15 minutes from two
types of users: {\em control}, or users who pay and use a high capacity
access link (105 Mbps); and {\em treatment}, or users who pay for 105
Mbps but were actually offered a 250~Mbps access link {\em without their
  knowledge}.  We evaluate three months of traffic demand for more than
6,000 Comcast subscribers, 2,219 of whom were in the treatment group.
We find that subscribers who are already using most of their available
capacity at the 105~Mbps service tier do not use significantly more
capacity at the higher service tier.  On the other hand, subscribers who
exhibit more moderate traffic demands often exhibit a significant
relative increase in their traffic demands.  This result suggests that
that even users who are not fully exhausting the available capacity at
one service tier may increase usage at higher service tiers, since the
improved performance at the higher tier may cause these subscribers to
use the Internet more than they otherwise would. We also observed that
the most significant increases in per-subscriber traffic demand as a
result of the upgrade occurred during non-prime-time hours on weekdays,
suggesting that this demographic of consumer may disproportionately
include users who work from home.  Such a phenomenon is also consistent
with with our observation that traffic demands at these higher service
tiers consistently rises throughout the course of the day, with no
mid-afternoon drop in traffic volume, as is evident in other studies.

% + prime time hours were 8-12 instead of 7-11 for high tier dataset
% + asymmetry?

The rest of the paper is organized as follows. In \autoref{sec:related} we 
overview some previous studies of traffic demand and service capacity. Then, in 
\autoref{sec:data}, we offer details about our data, sanitization, and 
characterization. We then proceed by describing our evaluation criteria and 
analyze traffic demand in response to a service tier upgrade in 
\autoref{sec:analysis}.
We summarize our findings in 
\autoref{sec:conclusion}.
%\sgfoot{We do this by studying spacio-temporal usage patterns in terms of peak 
%utilization, prime-time ratio, asymmetry, and prevalence.}
% + user taxonomy in discussion?
