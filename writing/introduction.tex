\section{Introduction}\label{sec:introduction}

PARA 1: SEE OTHER USAGE CASE STUDIES ABOUT IMPORTANCE\\
(importance of usage. important for fcc. important for isps capacity planning. bottlenecks to usage can be the user themselves, the transit, or the service. how the fcc is considering usage based benchmarks so its important to understand who the heavy users are and when do they consume the most \sg{should this fcc be in the last intro para or the discussion section?})

PARA 2: lower tier has been studied (dasu, africa, etc). but higher tier what happens? what happens when people are already satisfied?\\
(previous study shows usage is tough to determine. choices are random too many variables [36]. dasu guys study usage with many factors by monitoring transfer bytes throughout the world. designed as a natural exp they show that apart from capacity, price and performance (latency pack loss) also play an imp role in determining usage. the law of diminishing returns when accounting for other factors.)

PARA 3: Our study\\
(we study the change in usage of subscribers who do not need any more bb are offered more without their knowledge by controlled exp. these 2219 dev are selected randomly from the same locations and same tier and ISP (keeping price and performance latency and plots similar). Compared with 18355 devices sharing the same demographics but getting what they paid for, hopefully their demands and choices are actually similar due to such control. Allows an ISP to plan for capacity based on trends in usage demand - if demand don't increase users are satisfied with high tier.)

PARA 4: Interesting results and contributions\\
(Our contributions show the complicated nature of individual usage and capacity, and motivate the need of demand based tiers and benchmarks in the future. We see off peak usage increases for users not utilizing the link anyway. We also seee that in our high speed tier dataset, prime time hours become late (8-12). Also peak usage increases for low util users daily. In the dataset over three months only 11 users out of 1500 actually went over 105 Mbps.) \sg{maybe just interesting results and not contributions}

PARA 5: Higher perspective of interesting results and a line on policy makers\\
(The analysis of usage behavior to discuss different perspectives of bb utilization. We see that greedy rich users do not increase their usage so isps may conclude that they are satisfying their customers fully. But we also see that frugal rich users increase usage, especially in off peak, so conclude that as a consumer or policy maker capacity is still impacting perf. Both parties may have opposing views but look at the same observation. Motivates further study of the relationship between bb and capacity, esp. with similar control exp, and find out why this increase happened? Also look into usage demands as a metric used to determining broadband benchmarks.)

PARA 6: Roadmap of the paper\\
(section 2 background on previous usage work and industrial trends. section 3 data characterization and sanitization. section 4 empirical analysis looking at usage, peak usage, prime time, asymmetry, persistence and prevalence \sg{talk again a bit about usage - most interesting is how many did NOT change behavior}. section 5 is discussion on the differing perspectives of utilization (fcc vs isp) \sg{and different types of users taxonomy.} and concluding on how fcc considering usage definitely needs more input from our community)