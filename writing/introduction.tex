\section{Introduction}\label{sec:introduction}

With the large impact of broadband Internet on our daily lives and its
rapid increase in bandwidth-intensive services, policymakers and service
providers (ISPs) are trying to determine how much bandwidth consumers
need. With the proliferation of high quality video content, and the
recent boom in Internet-enabled consumer device, it is worth
studying---and continually re-evaluating---whether (and how) users
consume the capacity that ISPs offer.  Up to a certain point, users will
exhaust available capacity, and they will also adapt when more capacity
becomes available; this increased demand in turn drives provisioning.
Above certain speeds, however, the typical user no longer exhausts the
available capacity. At what speed does this inflection point occur?  How
do users adapt their demands when an ISP offers faster speed tiers?
Answers to these questions will ultimately help inform policymakers and
ISPs determine how to make investments in infrastructure, and when to
make them.

% + more work @ home
% + fcc wants to add usage as a bb parameter

Previous work that has studied the complex relationship between demand and capacity  
discovered that users who are already maximizing their usage on a given access link 
will continue to do so when they are migrated to a higher service
tier~\cite{dasu-imc2014}. In this paper, we study how the traffic demands of subscribers who are already 
on service plans with high downstream throughput respond to 
an undisclosed service plan upgrade as part of a randomized controlled trial 
(RCT). This experiment offers the unique opportunity to explore the
effects of a service-tier upgrade on 
user traffic demand while mitigating the cognitive bias of the service-tier upgrade 
by withholding that information from subscribers. To the best of our knowledge, 
this is the first such comparative study of usage behavior in a controlled 
experiment to study responses to service upgrades.

Our study is based on data collected from the residential home gateways
of Comcast subscribers in Salt Lake City, Utah. To measure traffic
demand, Comcast collects
aggregate byte counts every 15 minutes from two types of
users: {\em control}, or users who pay and use a high capacity access
link (105 Mbps); and {\em treatment}, or users who pay for 105 Mbps but
were actually offered a 250~Mbps access link {\em without their
  knowledge}.  We evaluate three months of traffic demand for more than
20,000 individual subscribers, 2,200 of whom were in the treatment
group. Our analysis shows that while the aggregate and mean daily usage
did not change significantly due to the upgrade, the median subscriber
increased traffic demands by about 20\%
during off-peak hours. Subscribers who are already using most of their
available capacity in 105~Mbps, do not use significantly more capacity
when they are moved to a higher tier, but subscribers with lower traffic demands
increased their daily 95th-percentile usage by about
10\% on any given day. More than than 99\% of subscribers did not
exhaust capacity for a sustained 15-minute interval any time during the
three months of our study.
% + prime time hours were 8-12 instead of 7-11 for high tier dataset
% + asymmetry?

Following the Federal Communications Commission (FCC) recent interest in
adopting usage (traffic demand) as one of the broadband benchmarks, we
discuss the possible implications of our observations on traffic demand
for both policymakers, who determine broadband deployment policies for
the US based on consumer demands, as well as ISPs such as Comcast, who
plan broadband deployment and capacity upgrades based on trends in
aggregate usage behaviors. We conclude that both policymakers and ISPs
may have different perspectives of the same observation, and motivate
the need to further study traffic demand as a broadband benchmark
metric.

The rest of the paper is organized as follows. In \autoref{sec:related} we 
overview some previous studies of traffic demand and service capacity. Then, in 
\autoref{sec:data}, we offer details about our data, sanitization, and 
characterization. We then proceed by describing our evaluation criteria and 
analyze traffic demand in response to a service tier upgrade in 
\autoref{sec:analysis}.
Lastly, we comment on the differing perspectives of traffic demand in 
\autoref{sec:discussion}, and summarize our findings in 
\autoref{sec:conclusion}.
%\sgfoot{We do this by studying spacio-temporal usage patterns in terms of peak 
%utilization, prime-time ratio, asymmetry, and prevalence.}
% + user taxonomy in discussion?
