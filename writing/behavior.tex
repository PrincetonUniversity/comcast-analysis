\subsection{Traffic Demand Per Subscriber}\label{subsec:behavior}

To characterize the diurnal traffic demand observed at the ISP, we first 
calculate traffic demand per subscriber (table \ref{tab:eval-criteria}), and 
then plot the median and 95\%-ile of total usage over a week for both 
\treatment{} and \control{} sets (figure \ref{fig:TS-data-rate-daily}).

We calculate demand by averaging the 
bytes transferred in uplink or downlink direction over the sample period (15 
minutes).\todo{ This allows us to capture both the average per subscriber 
demand at 
any time in a day, and the aggregate demand at the ISP over a longer period, 
such as a day or a week.}s


\begin{figure}[ht]
\begin{minipage}{\linewidth}
  \centering
  \includegraphics[width=\linewidth]{figures/describe-total-throughput-per-day[replace].png}
  \caption{agg (days) over means (devices): aggregate has no trough, peaks in the evening hours}
  %http://riverside.noise.gatech.edu:8083/separated/full/describe-total-throughput-per-day.png
  \label{fig:TS-data-rate-daily}
\end{minipage}
\end{figure}

\todo{replace fig 2 with gen weekday and weekend plot in total bytes not kbps}
\todo{ - mean demand, 95\% demand numbers during xx hours: just give ratio 
(avg) in prime-time and non-prime-time and say factor of increase xxx (don't 
write times) }
\todo{point: there is something happening in off-peak hour not in peak hours}
Figure \ref{fig:TS-data-rate-daily} shows the aggregate data rate for a day. 
We observed that the median traffic demand during 7:00 PM -- 7:00 AM is 
\todo{XXX} for both \treatment{} and \control{}. However, during off peak 
daytime (work) hours, between 7:00 AM -- 7:00 PM, the \treatment{} group has a 
median of \todo{XXX}, 20\% higher than the \control{} set.

\todo{- point: weekend and weekday look v different plateaued and no troughs}
We observe that the rise to the peak prime time hour usage 
on weekdays is not plateaued like the pattern observed on weekends (and 
holidays). A generic (median) weekday aggregate usage consists of a rise in 
usage that starts early in the morning that builds up to the prime-time period, 
peaks, and then falls sharply. We do not observe a trough in mid afternoon 
(between 2:00 PM -- 6:00 PM), as is usually the case for overall usage observed 
at US Fixed access providers \cite{sandvine20141h}.


% EXPLAIN:
% LACK OF TROUGHS: users' behavior in higher tier bandwidth
% DISCREPANCY IN DAYTIME OFFPEAK: ???   


\begin{figure*}[ht]
\begin{minipage}{1\linewidth}
\centering
%
\begin{subfigure}[b]{0.33\linewidth}
\includegraphics[width=\linewidth]{figures/cdf-all-bytes.png}
               \caption{Overall traffic demand per subscriber per 
sample\label{fig:CDF-data-rate}}
\end{subfigure}
%
\begin{subfigure}[b]{0.33\linewidth}
\includegraphics[width=\linewidth]{figures/cdf-max-per-device.png}
               \caption{Maximum and peak traffic demand per 
subscriber\label{fig:CDF-data-rate-max}}
\end{subfigure}
%
\begin{subfigure}[b]{0.33\linewidth}
\includegraphics[width=\linewidth]{figures/cdf-max-per-day-per-device.png}
               \caption{Daily maximum and peak 
traffic demand per subscriber.\label{fig:CDF-data-rate-max-daily}}
\end{subfigure}
%
\end{minipage}
\caption{Traffic Demand: The overall data rate per subscriber per sample is the 
same for both control and treatment, but the maximum (and 95 percentile) data 
rate per subscriber varies for test and control, and this variation occurs 
daily as well as over the whole lifetime. 50 percent of the users with the 
heaviest utilization in treatment and control have the same traffic demand, 
but 50 percent of the users with light utilizations in the treatment have 
higher demands than the 50 percent of light users in control.}
\label{fig:traffic-demand}
\end{figure*}


A comparison of the aggregate traffic demand distribution of all devices 
and sample periods shows that the \treatment{} has no significant affect on the 
mean traffic demand. 
