\subsection{Traffic Demand Per Subscriber}\label{subsec:behavior}

\begin{figure}[t]
\begin{minipage}{1\linewidth}
\centering
%
\begin{subfigure}[b]{.99\linewidth}
\includegraphics[width=\linewidth]{figures/weekday_demand_mean.pdf}
               \caption{Weekday traffic demand\label{fig:weekday-daily-usage}}
\end{subfigure}
%
\begin{subfigure}[b]{.99\linewidth}
\includegraphics[width=\linewidth]{figures/weekend_demand_mean.pdf}
               \caption{Weekend traffic demand\label{fig:weekend-daily-usage}}
\end{subfigure}
%
\end{minipage}
\caption{Average subscriber demand (bytes every 15-minutes)}
\label{fig:traffic-demand-timeseries}
\end{figure}

Figure \ref{fig:traffic-demand-timeseries} shows the downlink traffic demand (bytes)
of an average subscriber over a 15-minute measurement period in a week.
We observe that subscriber behavior differs
significantly on weekdays and weekends. On weekdays, traffic demand 
increases monotonically from morning until prime-time in the evening. On 
weekends, there is a sharp rise in demand in the early morning period. Then, the
demand plateaus until the next sharp rise during the evening prime-time hours.
Previous reports indicate that the aggregate traffic volume for US fixed access
link providers usually troughs during mid-afternoon hours (between 2:00 PM -- 6:00 PM)
~\cite{sandvine20141h}. We do not observe such a trough in the subscriber 
demand in our dataset.

\begin{figure}[t]
\begin{minipage}{1\linewidth}
\centering
%
\begin{subfigure}[b]{1\linewidth}
\includegraphics[width=\linewidth]{figures/cdf-all-bytes.pdf}
               \caption{Overall traffic demand for all subscribers at all times\label{fig:CDF-data-rate}}
\end{subfigure}
%
\begin{subfigure}[b]{1\linewidth}
\includegraphics[width=\linewidth]{figures/cdf-per-device-perc95.pdf}
               \caption{Peak (95\%) traffic demand per subscriber\label{fig:CDF-data-rate-perc95}}
\end{subfigure}
%
\end{minipage}
\caption{Traffic demand (bytes every 15-minutes) for \control{} and \treatment{}\label{fig:traffic-demand-cdf}}
\end{figure}

\begin{figure}[t]
\centering
\includegraphics[width=\linewidth]{figures/diff_perc95_bytes_subsc-overall.pdf}
               \caption{Difference between \treatment{} and \control{}
               in peak (95\%) traffic demand\label{fig:diff-perc95}}
\end{figure}

Figure \ref{fig:CDF-data-rate} shows the distribution of the
all bytes transferred in the \treatment{} and \control{} set
over the three months of the dataset.


Figure \ref{fig:CDF-data-rate-perc95}  plots the distribution of the
peak (95th percentile) demand of the subscriber. We observe that the maximum demand increases 
due 
to the service upgrade throughout the \control{} group. However, the increase is 
higher for
subscribers who have a low traffic demand.

We see that 30\% of the households from the \test set have a low 
peak demand (under 0.1 Mbps), while 40\% of the \control set households 
are under 0.1 Mbps. Thus, the absolute peak demand does not increase when 
compared to the access link capacity, but there is certainly an increase in 
peak demand of devices that had a low requirement, due to the change in 
capacity.

On investigating further, we observed that subscribers with low demand traffic 
increased their demand daily. There is negligible change in the daily peak 
demand for users
who have high demands, and a large increase for users with a low demand.
Furthermore, we observe this affect is also present in the uplink.
There could be many reasons for this increase in 
demand, such as short term activities (short videos or web browsing) 
that have a slightly higher traffic demand. Studying the applications 
responsible for such behavioral changes in traffic demand is out of the
scope of this paper and we leave it to future work.