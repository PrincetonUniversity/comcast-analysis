\subsection{Peak Traffic Demand}\label{subsec:peakratio}

The Sandvine Reports show that although the mean usage has remained
stable for the past few years, usage during peak-times has increased
drastically~\cite{sandvine20141h}. To measure this growth, they introduce the
concept of peak period, measured when the network is within 95\% of its highest 
point. Although, these reports present a good view into aggregate usage 
patterns over a month,
they neglect to analyze usage characteristics individually. Inspired by their
definition, we measure the disparity between the 95 percentile of the peak and 
median
usage of each household within a day, and call this the \emph{Peak-Ratio}.

We study the highest traffic demand seen by a household both in its lifetime, 
and on each day. We define the daily peak traffic demand per household as the 
95 percentile demand of the household per day, and the overall peak traffic 
demand per household  as the 95 percentile demand of the household over the 
three month dataset.

Figure~\ref{fig:CDF-data-rate-max} shows a distribution of the maximum traffic 
demand over the lifetime of a subscriber. \dots

Surprisingly, we see that 30\% of the households from the \test set have a low 
peak demand (under 0.1 Mbps), while 40\% of the \control set households 
are under 0.1 Mbps. Thus, the absolute peak d does not increase when 
compared to the access link capacity, but there is certainly an increase in 
peak demand of devices that  had a low requirement, due to the change in 
capacity.

On investigating further, we observed that subscribers with low demand traffic 
increased their demand daily. This is similar to the behavior observed in 
figure~\ref{fig:traffic-demand-timeseries}, showing that the traffic during 
prime-time remains similar, but lower demand throughout the day  is higher for 
the \test. We speculate that there could be many possible reasons for this 
increase in demand, such as short term downloads (short videos, web browsing, 
etc.) achieving a slightly better data rate on a small time scale. Studying this 
behavioral change would require data used per application in each time period 
and we leave it to future work.

\begin{figure}[t]
\begin{minipage}{1\linewidth}
\centering
\includegraphics[width=1\linewidth]{figures/peakratio-CDF-devices-MEAN.pdf}
\caption{Distribution of peak ratio for subscribers in the treatment and 
control groups. For 40 percent of the users with most disparity between their
95 percentile and mean demand in a day, the peak-ratio stays the same due to 
the service upgrade. For 60 percent of the users with a low peak ratio in the 
control group have a higher peak ratio in the treatment group. \red{fix aspect 
ratio, not log x scale}}
\label{fig:CDF-peak-ratio-mean}
\end{minipage}
\end{figure}

To further characterize and compare the deviation of data rate for the \control 
and \test set, we examine \emph{peak-ratio} as defined above. 
Figure ~\ref{fig:CDF-peak-ratio-mean} shows that the mean peak-ratio for each 
device in the \test set is much larger than that of the \control set.
\todo{replace much larger with the exact number or percentage}.
\sg{Taken together} with our observations of a lower prime-time ratio of the  
\test set (section~\ref{subsec:primetime}) this implies that there are 
households in the \test set that achieve a peak-ratio $>$ 1, but not during the 
prime-time hour. We believe that these households might actually be small 
businesses or work-at-home users that peak during daytime hours instead of 
evening hours.

The mean peak-ratio varies between \todo{X} and \todo{Y}. There are 
some households that have a very even usage throughout the day (low peak ratio), 
and others that are extremely aggressive only at a certain times of a day (high 
peak ratio).