\subsection{Peak Traffic Demand}\label{subsec:peakratio}

We study the highest traffic demand seen by a household both in its lifetime, 
and on each day. We define the daily peak traffic demand per household as the 
95 percentile demand of the household per day, and the overall peak traffic 
demand per household  as the 95 percentile demand of the household over the 
three month dataset.

Figure~\ref{fig:CDF-data-rate-max} provides a distribution of the highest 
average data rate a household achieves. To avoid outliers, we also plot the 
95\%-ile of the max data rate achieved by households in both \test and \control 
sets. We see that a median household is expected to achieve the highest data 
rate of between 1 -- 10 Mbps over its lifetime. This is much lower than the 
access link capacity, indicating that the median device has a utilization ratio 
(avg data rate:capacity) under 0.1 in our dataset. The number of households 
that increased their peak demand beyond the \control set's 105 Mbps 
capacity were negligible (see \autoref{subsec:prevalence})

Surprisingly, we see that 30\% of the households from the \test set have a low 
peak demand (under 0.1 Mbps), while 40\% of the \control set households 
are under 0.1 Mbps. Thus, the absolute peak d does not increase when 
compared to the access link capacity, but there is certainly an increase in 
peak demand of devices that  had a low requirement, due to the change in 
capacity.

To investigate this further, we also study the \emph{peak demand per device 
on a daily basis}. Figure~\ref{fig:CDF-data-rate-max-daily} shows that for 30\% 
of the devices, the maximum data rate in the \test set is consistently higher 
than the \control  set, although it is no where near the actual access link 
capacity.

This is similar to the behavior observed in figure~\ref{fig:TS-data-rate-daily}, 
showing that the peak usage during prime-time is unaffected, but lower 
demand throughout the day  is higher for the \test. We speculate that 
there could be many possible reasons for this increase in demand, such as 
short term downloads (short videos, web browsing, etc.) achieving a slightly 
 better data rate on a small time scale. Studying this behavioral change 
would require data used per application in each time period and we leave it to 
future work.


\paragraph{Peak Ratio: }The Sandvine Reports show that although the mean usage has remained
stable for the past few years, usage during peak-times has increased
drastically~\cite{sandvine20141h}. To measure this growth, they introduce the
concept of peak period, measured when the network is within 95\% of its highest point.
Although, these reports present a good view into aggregate usage patterns over a month,
they neglect to analyze usage characteristics individually. Inspired by their
definition, we measure the disparity between the 95 percentile of the peak and 
median
usage of each household within a day, and call this the \emph{Peak-Ratio}. In
section~\ref{subsec:peakratio} we show that the peak ratio can be used
to divide users in the same tier based on their usage behavior.

% the ratio of the 90\%-ile to the median throughput per day.


\begin{figure}[ht]
\begin{minipage}{0.90\linewidth}
\centering
\includegraphics[width=1\linewidth]{figures/peakratio-CDF-devices-MEDIAN.png}
\caption{Median peak ratio per device showing that test set has higher daily ratio (50 times by median). Thus ISPs should condition their networks to 50 times the median usage for each user added in the worst case scenario.}
%http://sites.noise.gatech.edu/~sarthak/files/comcast/plots/full_dw/peakratio-CDF-devices-MEDIAN.png
\label{fig:CDF-peak-ratio-median}
\end{minipage}
\end{figure}

To further characterize and compare the deviation of data rate for the \control and \test set, we examine \emph{peak-ratio} as defined above. 
Figure ~\ref{fig:CDF-peak-ratio-median} shows that the median peak-ratio for each device in the \test set is much larger than that of the \control set.
\todo{replace much larger with the exact number or percentage}.
\sg{Taken together} with our observations of a lower prime-time ratio of the \test set (section~\ref{subsec:primetime}) this implies that there are households in the \test set that achieve a peak-ratio $>$ 1, but not during the prime-time hour. We believe that these households might actually be small businesses or work-at-home users that peak during daytime hours instead of evening hours.

The median peak-ratio per device itself shows a large range, from 1 to 10e6 (figure~\ref{fig:CDF-peak-ratio-median}), and the maximum peak-ratio per device was an order higher. Clearly there are some households that have a very even usage throughout the day (low peak ratio), and others that are extremely aggressive only at certain times (high peak ratio).


% other results:
%big difference (2 x median ratio) in per device per day ratios of 90%ile:median.
%weird shape again for values < ratio 100
%big difference in this ratio per day, and it is consistent across all individual sets + months.
%very large for Dec, slightly smaller for Nov
%interestingly, at higher ratios control is slightly > test. This means that certain devices in control set have a huge std (ratio) in a day as compared to test set which has a lower “max” ratio.
