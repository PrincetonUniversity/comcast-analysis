\section{Methodology}S
\label{sec:methodology}
\todo{this explains our methodology}

% not telling users that the connection was upgraded lets us study correlation between usage and capacity without the bias of user changing his behavior due to external factors (buying a device, being more aggressive as they bought a new plan)




% DASU supports correlation by measuring usage on each ISP tier, even for those who changed tiers, but that is biased by user buying a higher tier when they feel unsatisfied. It does not have a good representation of the case when the ISP offers extra bandwidth without charging. Even if it does, the user knows about the capacity bump inducing a change in behavior. Our dataset is unbiased in that manner, we don't expect users to know they have a higher capacity and still study if there is change in behavior


% explain that due to no bias in city, and large number of users, this can be interpretted as change or no change in behavior due to increase in capacity of the link. We expect the baseline behavior of these users to be exactly the same. Thus we will put our questions as "change in behavior due to increase in capacity" rather than analysis of two different datasets as we have attempted to eliminate all biases.


% we analyze based on three criteria
% usage patterns
% peak utilization
% prime time ratio
% asymmetry in data
% user taxonomy



% importance of measuring peak



% based on our results we propose a metric peak ratio to segregate users
% peak ratio cdf vs no of devices
% peak ratio cdf vs time of day where peak occured
% no of devices cdf vs time of day where peak occured


% the segregation splits business and casual users based on usage and motivates our discussion on multiple benchmarks


% shortcomings of data + advantages (user end instead of sandvine which is core internet?? so we can comment on per user patters??)