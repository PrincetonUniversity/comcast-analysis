\section{Methodology}
\label{sec:methodology}

We specify the evaluation criteria we use to compare usage 
patterns of the \test 
and \control sets.

In the following, we define capacity as the access link capacity offered by the 
ISP to the user. In the case of the Comcast dataset, capacity is 105 Mbps for 
the \control set and 250 Mbps for the \test set. Similarly, we define usage as 
the actual traffic (bytes uploaded or downloaded) by the household being 
studied. We use usage across 15 min time slots to derive the average data rate 
and compare it to the capacity in our analysis.

\todo{This section is not complete}

\subsection{Evaluation Criteria}

We analyze based on three criteria:
\begin{itemize}
\itemsep0em 
\item usage patterns
\item peak utilization
\item prime time ratio
\item peak ratio
\item traffic asymmetry 
\end{itemize}



\paragraph{The Importance of Measuring Usage at Prime Time:}
\begin{itemize}
\itemsep0em 
\item Capacity planning is concerned about “prime-time” and peak behavior, i.e., max data rate per device is more important than the average rate of that device.
\item Idle hour behavior is expected to stay the same, regardless of access link
\item peak utilization definition
\end{itemize}


\paragraph{Prime-Time Ratio} definition of prime time
\begin{itemize}
\itemsep0em 
\item Sandvine defines Network Prime-Time ratio to measure the concentration of network usage during the prime-time evening hours.
\item FCC says prime-time is 7-11 PM.
\item Prime-Time ratio = absolute levels of network traffic during an average peak period hour with an average off-peak hour.
\item Measure PT = avg data rate during a peak hour period : off-peak period.
\end{itemize}

 Previous studies and reports ~\ref{sandvine} have indicated that peaks in prime-time occur mainly due to downlink video traffic, and may result in higher latency.

\paragraph{Peak Ratio}
\begin{itemize}
\itemsep0em
\item define to measure imbalance in a day
\item for each device,peak ratio  =    90-\%ile data rate per day / median data rate per day
\item this metric can also be used to segregate users, based on who peaks at what time [section ~\ref{sec:discussion}]
\end{itemize}

%\subsection{Limitations}
%shortcomings of data + advantages (user end instead of sandvine which is core internet?? so we can comment on per user patterns??)