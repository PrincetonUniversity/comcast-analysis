\section*{Appendix}
\label{sec:appendix}

In the FCC report No. 14-113, released on Aug 5, 2014~\cite{fcc2014progress-report},
the Commission asks some relevant questions about broadband usage, and requests comments
from the community to improve its decision making process. We summarize their hypothesis
and comments with regards to speed benchmarking as follows\footnote{Note that the issue
\# corresponds to the paragraph in the Eleventh Broadband Progress Report, FCC No. 
15-10A1 ~\cite{fcc2015progress-report}\label{foot:fcc-issue-numbers}}

% comments for FCC:
% Can be mitigated by asking for usage instead from ISP first hop per household byte counters every 15 min. At least estimate usage (even if many apps in parallel) do they bottleneck or no?
% Why our comments are important:
%- measurement guys, well informed, not frustrated laymen, understand the need for reco committee.
%- not biased (ATnT, verizon etc).
%- also understand broadband dynamics and net neutrality
%- important issues deserving our timely attention

\subsection{Issues from the Tenth Broadband Progress Report}
\label{subsec:fcc2014}

Issues from the Tenth Broadband Progress Notice of Inquiry FCC 14-113, released August 5, 2014~\cite{fcc2014progress-report}

% purpose
\paragraph{3.} With this Inquiry, we start anew by analyzing current data and seeking information that will enable the Commission to conduct an updated analysis for purposes of its next report. In particular, we seek comment on the benchmarks we should use to define ``advanced telecommunications capability,'' explore whether we should establish separate benchmarks for fixed and mobile services, which data we should rely on in measuring broadband, whether and how we should take into account differences in broadband deployment, particularly between urban areas versus non-urban and Tribal areas, and other issues. We seek comment on whether we should modify the 4 megabits per second (Mbps) download and 1 Mbps upload (4 Mbps/1 Mbps) speed benchmark we have relied on in the past reports. We also seek comment on whether we should consider latency and data usage allowances as additional core
characteristics of advanced telecommunications capability.

% speed and uses
\paragraph{6. Broadband Speed}
Section 706 refers to consumers' ability ``to originate and receive high-quality voice, data, graphics, and video telecommunications using any technology.''
We seek comment on the
appropriate speed benchmark that would permit users to achieve the purposes identified in section 706. We ask parties to provide information on which of the applications described in section 706’s definition of advanced telecommunications capability
Americans are using most today and how they affect the
need for broadband services at a particular speed.14
Does service with speeds of 4 Mbps/1 Mbps provide
consumers the ability to originate and receive these services? For example, consumers increasingly use VoIP, social networking, video conferencing, and streaming video over their broadband connection.
In
particular, we have seen tremendous growth in the online video and audio markets in the past few years.16 In its most recent report, Sandvine, a company that researches global Internet trends, indicates that real- time entertainment, such as streaming video and audio, continues to be the largest traffic category on virtually every network.17
Sandvine adds that real-time entertainment ``is responsible for over 63\% of
downstream bytes during peak period.''18
Given the demand for video services and the introduction and
use of new services on the market, the Commission may find that the 4 Mbps/1 Mbps speed benchmark no longer allows consumers the ability to ``originate and receive'' the broadband services identified in section 706. We seek comment on whether we should continue to benchmark broadband based on actual speeds, rather than advertised speeds, to the extent the two are different.

% multiple benchmarks?
\paragraph{7. } Below, we seek comment on ways the Commission could determine speed requirements,
and in particular seek comment on assessing common household broadband uses or relying on broadband adoption rates as bases for establishing a speed benchmark. We also seek comment on adopting multiple speed benchmarks.

% peak usage time
% Peak usage is defined as the average data consumed between 7:00 to 11:00 pm on weeknights. The FCC asks if evaluating peak time usage is an efficient metric, or if average usage over a day should be considered instead.
\paragraph{13. \emph{Relying Upon Peak Usage Time.}}
 We seek comment on whether we should assess bandwidth requirements for a typical household during peak Internet usage periods, from 7 pm to 11 pm on weeknights.
We seek comment on the types of broadband uses that are common, often used simultaneously, within a household during peak periods, and the appropriate bandwidth that would be necessary to accommodate those uses, to the extent that we determine that it is reasonable under section 706(b) to consider multiple simultaneous uses of broadband. We also seek comment on whether it is reasonable under the statute to set a speed benchmark on the basis of ``peak usage time.'' Is peak usage time an efficient metric? Should we instead consider the average household usage over a 24-hour period or over some other time period, or in some other manner entirely? We recognize that every household is unique, and that the services each household member uses will vary. We seek comment on whether establishing a reasonable household usage scenario during peak periods will assist the Commission in identifying a benchmark that is a necessary component of ``advanced telecommunications capability.''

% upload and download speeds needed at peak periods
\paragraph{14. } In light of the FCC 2011 Household Broadband Guide and the FCC 2014 Household
Bandwidth Scenarios, we seek comment on whether the Commission should adopt a higher download speed benchmark, such as 10 Mbps, to more appropriately reflect the statutory requirements in section 706.34
According to the FCC 2011 Household Broadband Guide, service meeting a 10 Mbps download benchmark would fall within the mid-range needed by a three-user household with moderate broadband use, but would not accommodate demand for a three-user household with high use.35
The FCC 2014
Household Bandwidth Scenarios suggests that a 10 Mbps download speed could accommodate a ``Moderate Use Household,'' including allowing a family of three at peak periods to stream a movie, participate in online education, surf the web, and have a mobile device syncing to its email account.

\paragraph{15. }
We seek comment on whether a download speed of 10 Mbps would adequately reflect
Congress’s goal of evaluating advanced telecommunications capability.37
Does 10 Mbps satisfy current
demand, especially during peak time? Even assuming that it does, should the benchmark be higher than the minimum necessary to meet existing demand, i.e., should the benchmark be set to accommodate some level of anticipated future demand, particularly if the Commission does not intend to adjust the benchmark annually? Some forecasts of broadband household needs suggest a higher download speed may be necessary.38
For example, would a significantly higher download speed, such as 15 or 25 Mbps,
more accurately fulfill Congress’s intent? How should the Commission forecast future household broadband uses to justify such a benchmark?

\paragraph{16. }
We seek comment on whether a 1 Mbps upload speed will suffice to meet the
requirements set forth in section 706. T

\paragraph{17. }
Other data suggest that it might be appropriate for the Commission to increase the upload speed benchmark for purposes of addressing the statutory requirements in section 706
increasingly use interactive real-time services and upload content, such as pictures, documents, and engage in video calls.43
How should we consider the use of symmetrical services, such as two-way video
calling; the uploading of media to social networks; and cloud storage? Which do consumers use more – standard or HD video calls? Today, consumers can have a real-time video consultation over their broadband connection at home with doctors many miles away and this type of service may require higher upload speeds.44
In addition, some states have adopted a 1.5 Mbps upload speed as a benchmark.45 We
thus seek comment on whether a 1 Mbps upload speed is sufficient to meet the goals set forth in section 706. We also seek comment on whether there is a basis for the disparity between download and upload speeds in any speed threshold(s) used by the Commission. For example, if the Commission increases the download speed benchmark, should it also increase the upload speed benchmark?46
Why or why not?

% Adoption based broadband benchmarking
\paragraph{18. Setting a Speed Benchmark Based on Adoption Rates.} We seek comment on whether the Commission should consider the rates at which consumers are adopting particular speeds when setting a speed benchmark. We seek comment on whether a higher benchmark is appropriate when the Fourth Measuring Broadband America Report indicates that consumers continue to migrate to higher broadband speeds.47
In setting a speed benchmark, should we consider the speeds available in urban areas, as compared to the speeds available in other areas, and if so, how should we take any disparities into account?48
How should we consider that one report indicates that the average connection speeds in nine (Continued
countries are higher than the United States’ average speed or that the average connection speed in the United States is almost three times the global average when setting a speed benchmark in the next report?

% Taxonomy of users even in a single speed tier
\paragraph{19. } Should the benchmark be based on the fastest speed tier to which a substantial portion of
consumers subscribe? How should the Commission define ``substantial portion'' and how should we interpret such demand? Would using such a metric accurately reflect the market choices and needs of consumers based on the service offerings available to them? Does a particular adoption rate (to be determined) by consumers with access to broadband demonstrate that there is sufficient demand for that speed tier? Does adoption at a certain speed demonstrate or suggest that service of that speed is necessary to enable users to originate and receive high-quality voice, data, graphics, and video telecommunications? How should we account for the fact that higher speed services may not be offered in parts of the country? Should we look exclusively at adoption rates in areas where a given speed has been deployed if we select an adoption-based benchmark?

% Adoption based contd...
%The FCC's previous experience suggests that broadband benchmarks should be based 70\% adoption rate to encourage providers to increase broadband penetration. This would motivate a benchmark of 1 Mbps, however, the uplink speed benchmark has been set to 3 Mbps. Does this aggressive limit adequately anticipate the increasing use of symmetrical services, such as two-way video calling?
\paragraph{20. } If we were to set our benchmark based on adoption, the Commission would need to
determine what adoption rate would be necessary to ensure the speed benchmark was reasonable. For example, the Bureau assumed a subscription rate of 70 percent in modeling the costs of deploying broadband to rural America, although we note that was in a different context, and represents a modeling assumption rather than a substantive determination.51
Would a benchmark based on a 70 percent adoption
rate ensure that our benchmark is reasonable, attainable, and sustainable? Would a lower or higher benchmark be better and if so, why? If the median household chooses to adopt a speed tier, does that demonstrate that there is sufficient demand for that speed tier to suggest that all consumers should have the option of subscribing to it? What about a more forward-leaning adoption rate, such as 30 percent? Note that based on SBI Data and Form 477 Data, as of June 2013, 58 percent of households adopted fixed services of at least 3 Mbps/768 kbps, 47 percent of households adopted fixed services of at least 10 Mbps/768 kbps, 41 percent of households adopted fixed services of at least 10 Mbps/1.5 Mbps, 30 percent of households adopted fixed services 10 Mbps/3 Mbps, and 21 percent of households adopted fixed services of 25 Mbps/3 Mbps. To what extent should we evaluate the relationship between price and adoption and if so, how? For example, if a provider charges nominally more for a 25 Mbps download service than a 10 Mbps download service, consumers may adopt the higher speed service regardless of

% separate benchmarks: min broadband = 10/1, better broadband = 25/3, schools and libraries = 100...
%broadband requirements are not uniform throughout the nation. Some users will have significantly greater needs. The FCC is interested to know whether it should opt for multiple benchmarks depending on user scenario, usage, occupation, etc.? 
\paragraph{22. Multiple Speed Benchmarks.}  The foregoing discussion focuses on selecting a single
speed benchmark. We also seek comment on whether the Commission should consider establishing multiple benchmarks. Multiple benchmarks could improve our ability to assess whether advanced telecommunications capability is being deployed in a reasonable and timely manner by recognizing that broadband requirements are not uniform throughout the nation. For example, while the ``typical use'' 10 Mbps download speed benchmark described above is intended to satisfy common household broadband demand, some users, such as larger families or teleworkers, will have significantly greater bandwidth needs.52
We seek comment on whether the Commission should adopt more than one speed benchmark and if so, how we should use the different benchmarks to evaluate whether deployment is occurring in a reasonable and timely manner.

% IMP: Usage as a Broadband Benchmark
\paragraph{27. } The Commission has indicated that it might consider data usage allowance as a core characteristic that affects what consumers can do with their broadband service.64
Should we include
usage in our section 706 assessment? If so, how? We seek comment on what data usage allowances most broadband providers offer today, and the impact of these usage allowances on setting a benchmark. For example, do consumers routinely exceed the usage allowance for the service to which they subscribe and if so, is additional capacity available for an additional fee? If so, how frequently do consumers avail themselves of that option?

% how to measure the amount of data consumers use - form 477
\paragraph{29. }How would the Commission implement a broadband usage threshold? Should the Commission focus on the amount of data that consumers actually use each month, instead of what broadband providers typically offer? What information, reports, or other sources are available to measure the amount of data consumers use monthly? In particular, are there any sources concerning usage that the Commission could use to assess which carriers meet or do not meet the usage threshold?

% How should deployment be measured
\paragraph{37. }
We seek comment on ways to improve the Commission’s annual broadband progress reports, such as incorporating new data sources or conducting our analysis differently. Are there other ongoing efforts to collect broadband deployment or availability data that were not available, or that we did not include, in prior reports? We seek input on whether there are any particular surveys or other reports that would be particularly beneficial to our section 706 analysis.


\subsection{Issues from the Eleventh Broadband Progress Report}
\label{subsec:fcc2015}

Issues from the Tenth Broadband Progress Report and Notice of Inquiry FCC 15-10, released February 9, 2015~\cite{fcc2014progress-report}~\cite{fcc2015progress-report}

% why some people do not adopt broadband - motivation to study usage patterns
\paragraph{7. } As part of our inquiry into ``the availability of advanced telecommunications capability to all Americans,''14 we also examine broadband adoption—some reasons why Americans choose not to adopt broadband could reflect factors that are relevant to its ``availability,'' such as price and quality. While we continue to see that adoption lags behind deployment to a significant degree, at all speeds,15 we do not know precisely why. The recent 2014 NTIA Digital Nation Report found the top reason given for non-adoption was consumers simply not wanting broadband, and the second most cited reason was
because it was too expensive.16 Americans with lower median incomes and where the poverty rate, rural population rate, and unemployment rate is higher tend to have lower broadband adoption rates.17 These facts raise questions about whether broadband is ``availab[le] . . . to all Americans'' as the statute requires, and we will continue to evaluate both how we can improve our analysis in future Reports and how the Commission can address the adoption gap. Although we evaluate adoption separately from deployment and our determination about whether broadband is being deployed in a reasonable and timely fashion stands independently from our evaluation of broadband adoption, examining adoption is useful both as an indicator of what Americans may consider to be needed and separately as a stimulator for deployment.

% need to add usage patterns to this study
\paragraph{10. }
Future reports will benefit from analysis of more comprehensive and reliable data. In particular, once we begin to rely on the mandatory Form 477 data collection for deployment information, the reliability of the mobile and satellite data should improve substantially.25 In addition, we expect to examine other factors of availability, including usage allowances and price, latency, whether service at the relevant speed is available on a consistent and reliable basis, and whether the network is secure. We expect to examine these factors in the next Inquiry and will seek out ways to improve our ability to evaluate them, relevant to both fixed and mobile services.26

%deployment will occur only if people adopt and ISPs can make money off it => must study why all people don't adopt => study usage (bb not needed) vs price (bb too expensive).
\paragraph{12. }
As a consequence of our conclusion that advanced telecommunications capability is not being deployed to all Americans in a reasonable and timely fashion, section 706 mandates that the Commission ``take immediate action to accelerate deployment of such capability by removing barriers to infrastructure investment and by promoting competition in the telecommunications market.''28

% multiple standards - why and why not
\paragraph{23. } In the 2014 Broadband Progress Notice of Inquiry, we asked whether we should establish multiple benchmarks, including, for example, a ``forward-looking'' benchmark.
NCTA recommends that the Commission ``should use multiple benchmarks in performing its section 706 analysis rather than just one'' because doing so would ``better reflect the multi-faceted nature of today’s broadband marketplace, which features a wide variety of technologies and services that are able to cater to the varying needs of American consumers.''120
Others in the record disagree, stating, for example, that
``[t]here is no substantive rationale provided to warrant having multiple standards. One standard sends a clear and easily understood message to all parties.''121
At this time we decline to use multiple benchmarks
in this Report. We find that the single benchmark we use in this Report, which relies on a higher speed benchmark than prior reports, is the appropriate means to allow us to analyze whether advanced service is being deployed.

% speed factor
\paragraph{26. } In past Reports, the Commission has identified a speed benchmark against which to
measure broadband. The Commission has recognized that the benchmark must be periodically reassessed in light of market offerings and consumer demand.132 The 2010 National Broadband Plan recommended updating the broadband benchmark every four years.133 In the 2014 Broadband Progress Notice of Inquiry, we sought comment on whether we should update the 4 Mbps/1 Mbps broadband benchmark.134 The Commission updated the speed benchmark once before, in 2010, from 200 kbps/200 kbps to 4 Mbps/1 Mbps135 and we find it is time once again to update the speed benchmark.136 For purposes of this Report, we conclude that meeting the definition of ``advanced telecommunications capability'' requires consumers to have access to actual download (i.e., to the customer) speeds of at least 25 Mbps and actual upload (i.e., from the customer) speeds of at least 3 Mbps (25 Mbps/3 Mbps). For schools and classrooms, we use the same benchmark that the Commission already established for schools and classrooms of a short term benchmark of 100 Mbps per 1,000 students and staff and a long-term speed benchmark of 1 Gbps per 1,000 students and staff.

% consumers adopt higher speeds when they have the option
\paragraph{41. } Our own assessment of consumers’ likely needs is confirmed by examining broadband
adoption. When speeds of 25 Mbps/3 Mbps are available, a substantial and fast-growing number of consumers are adopting and migrating to higher speeds. Examining the adoption trends from December 2011 to December 2013, we find that the adoption rate of this service or higher quadrupled.196 Customers are deciding for themselves at a very rapid rate that they need services at this or higher speeds

% household level rather than user level granularity
\paragraph{47. }
Second, we consider what speeds are needed to ensure that consumers enjoy ``advanced''
capabilities. As noted above, households usually are comprised of two or more persons, and it is not uncommon for each person in the household to use more than one broadband device simultaneously.207 Because consumers usually purchase fixed broadband service for the household, and because the deployment data represent households rather than individuals, we find it reasonable to consider broadband needs at a household level, rather than what each individual household member, individually, may need.208 The record further supports a household analysis. The City of Boston, for example, states that the benchmark ``must be sufficiently robust to allow every member of a household to use multiple devices simultaneously.''

% why single benchmarks might not work - there are many types of users within the same tier
\paragraph{52. }
Some commenters urge us to retain the 4 Mbps/1 Mbps benchmark228 or increase the
benchmark to 10 Mbps/1 Mbps. ATnT asserts that section 706(b) is not ``myopically'' focused on the subset of consumers who are the heaviest users.230 We agree. Section 706(b) is not ``myopically'' focused on any particular subset of consumers. But it does require the Commission to assess deployment of advanced services that are capable of performing specific functions, and doing so at a high quality.

% availability and deployment
\paragraph{64. }
We affirm the Commission’s prior findings that, for the purpose of our analysis, the terms broadband ``deployment'' and ``availability'' are broader than mere physical presence of broadband networks.263 This interpretation is the most natural reading of section 706(b). Moreover, section 706 requires the Commission to conduct an inquiry into broadband ``availability'' and determine whether broadband ``is being deployed'' in a reasonable and timely fashion. The statute does not indicate that a determination about whether broadband is being deployed must consider a set of circumstances narrower than the ``availability'' inquiry. We continue to interpret ``all Americans'' as establishing the goal of universal broadband availability for every American. As such, our annual determination as to how broadband ‘‘is being deployed’’ is simply an assessment of how well we are progressing toward the goal of ``availability to all Americans.''

% Factors influencing adoption: availability, price, quality, ...
\paragraph{65. } Some commenters urge that we look only to physical deployment.264 Others recommend
that we continue to interpret these terms more broadly, and consider, for example, price and quality.265 We conclude that in determining whether broadband is ``being deployed to all Americans in a reasonable and timely fashion,'' we must look at a variety of factors that affect access to broadband. Congress did not define the terms ``deployment'' and ``availability'' in section 706(b), nor did it define the term ''served'' in subsection (c). However, Congress included the term ''high-quality'' in the definition of advanced telecommunications capability. This supports our finding, now as in past years, that we should not consider physical facilities without also considering service quality. As explained in the last Report, the legislative history further supports the view that Congress intended us to examine more than physical network deployment.266 Accordingly, as in prior Reports, our inquiry includes an assessment of a variety of factors indicative of broadband availability, such as price, quality, and adoption by consumers, as well as physical network deployment.

%VI. REMOVING BARRIERS & PROMOTING COMPETITION 141.
\paragraph{141. }
In light of our finding that advanced telecommunications capability is not being deployed to all Americans in a reasonable and timely manner, the Commission must ''take immediate action to accelerate deployment of such capability by removing barriers to infrastructure investment and by promoting competition in the telecommunications market.''491 In the last two Reports, the Commission found numerous barriers to infrastructure investment.492 In particular, the high costs associated with deploying and operating a broadband network coupled with low broadband adoption rates, present barriers.493 As we have done in the past, we will continue to work on removing barriers to infrastructure investment by identifying and helping to reduce potential obstacles to deployment, competition, and adoption—concepts that we continue to recognize are tightly linked.494 By taking steps to remove any barriers to the deployment of networks, the Commission can continue its efforts of ensuring that all Americans have access to affordable, high-quality broadband.

%usage of broadband is a barrier to deployment of higher speeds.
% measure usage of areas with no deployment (open data) - if it seems aggressive for a certain percentage of total people in that are => cost price analysis to deploy bb there. (almost the same as comcast's requirement to get at least 10 signatures in rural areas)
\paragraph{142. } Because providers will consider adoption rates when determining whether to build out facilities and offer service in a particular area, we consider barriers to adoption as well as deployment. The key barriers to deployment and adoption include: (1) costs and delays in building out networks; (2) broadband service quality; (3) lack of affordable broadband Internet access services; (4) lack of access to devices and other broadband-capable equipment; and (5) barriers to entry by potential competitors and, consequently, lack of competitive choice for consumers.
%only way is to show provider that on aggregate the deployment will be revenue generating => usage patterns to show that lower tiers are not satisfactory
%but we can't do this with current data unless comcast gives us low tier data!
