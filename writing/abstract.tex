\begin{abstract}
We present an analysis of usage patterns of Comcast subscribers whose service was increased from one high service tier to another. This study focuses on changes in usage behavior of two groups of users in the same city, a control group (105 Mbps tier link) and a treatment group (250 Mbps tier link). Members of the treatment group were randomly selected from the control group to receive 250 Mbps without their knowledge.

Previous work has shown that users who are already maximizing their usage on a given access link will continue to do so when they are migrated to a higher service tier. We study how users who are already on service plans with high downstream throughput respond when they are moved to a higher service tier without their knowledge. We find that subscribers who are already using most of their available capacity do not use significantly more capacity when they are moved to a higher service tier; in contrast, 50\% of the subscribers who have low traffic demand increase their 95 percentile peak usage by 10\% on upgrading the service plan. We also show that the median daily usage increases by 20\% during off-peak hours.
\end{abstract}
