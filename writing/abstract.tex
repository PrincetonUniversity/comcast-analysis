\begin{abstract}

The Federal Communications Commission (FCC) has recently taken major decisions in an attempt to 
regulate broadband in the US, creating disputes between itself and ISP providers.
One such policy was the radical increase of broadband speed benchmark, that motivated both parties
to investigate these regulations from different perspectives. This led to the Commission
publically requesting comments on issues pertaining to ``advanced telecommunication capabilities''
to design future benchmarks based on metrics other than broadband speed.

In this work, we provide a much required input to the FCC's open questions,
supported by our analysis of usage patterns, rather than just the FCC’s speed benchmarks.
We analyse users with high tier access links and motivate
the need of multiple benchmarks based on peak usage of different types
of users. We also show that beyond a certain speed, user behavior is not
significantly impacted by further increases in broadband capacity. Therefore,
motivating the need to define benchmarks based on metrics other than broadband
speed to truly offer ``advanced'' capabilities.

\end{abstract}
