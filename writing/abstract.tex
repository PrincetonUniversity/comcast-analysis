\begin{abstract}
The Federal Communications Commission (FCC) has recently taken major decisions
in an attempt to regulate broadband usage and deployment in the US. One such
policy is setting the broadband speed threshold to 25 Mbps/3 Mbps, based on
aggregate measurements performed throughout the country. Furthermore, the
Commission also requests comments on issues pertaining to defining ``advanced
telecommunication capabilities''  to design benchmarks based on metrics
other than broadband speed.

In this work, we provide a much required input to the FCC's open questions,
supported by measurements and analysis of usage patterns rather than just
speed benchmarks. We analyse users with high tier access links and motivate
the need of multiple speed benchmarks based on peak usage of different types
of users. We also show that beyond a certain speed, user behavior is not
significantly impacted by further increases in broadband capacity, thus
motivating the need to define benchmarks based on metrics other than broadband
speed to truly offer ``advanced'' capabilities.

$We show the importance of planning capacity and benchmarking
broadband during peak hours rather than aggregate behavior. 
that beyond a speed, the usage patterns from homes with high tier access link
are not substantially impacted by offering more capacity. 

\end{abstract}