\begin{abstract}

fcc sees bb adoption low because dont need. study usage to find out why?

previous studies have shown capacity and usage law of dimi. true for any low 
speed tier - one can always use a higher pipe. But what about people already on 
a higher tier, will usage keep increasing? How will ISPs deal with satisfying 
users if regardless of the capacity they can always do with more? How will the 
FCC define bb benchmarks when usage keeps being correlated with the capacity 
itself?

our question: what happens when people with thick pipes get more, without 
knowing. i.e, how far does the law of diminishing returns go for a group that 
is essentially satisfied.

do a controlled case study controlling as many factors as possible and upgrade 
high speed tier, 105 Mbps, to an even higher 250 Mbps to study usage patterns.

study results are uber cool. different from previous work the peak time usage 
may not be increasing but usage in off peak hours does! also 1\% of the 
group actually start using their connection aggressively, way more than the 105 
Mbps capacity over 15 min intervals, indicating long term prevalent use.



% 
% The Federal Communications Commission (FCC) recently declared that ``advanced
% telecommunication capabilities'' are not being deployed in the US in a timely
% fashion~\citep{fcc2015progress-report}. In an attempt to encourage Internet 
% Service Providers (ISPs) to invest in deploying advanced broadband, the 
% Commission has requested comments on issues pertaining to availability, 
% deployment, and adoption of broadband.
% 
% In this work, we provide a much required input to the FCC's open questions,
% supported by our analysis of usage patterns. We present case study
% examining the relationship between supply (availability) and demand
% (adoption) in a controlled experiment. Our analysis shows that peak user demand
% within a single high speed tier is highly diverse, even when accounting for
% factors such as capacity, cost, performance, and location. This work motivates
% the need to adopt user demand as a benchmark to further the deployment
% of ``advanced'' broadband capabilities in the US.
\end{abstract}
