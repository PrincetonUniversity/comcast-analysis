\begin{abstract}
  Internet service providers are facing mounting pressure from
  regulatory agencies to increase the speed of their service offerings
  to consumers; some are beginning to deploy gigabit-per-second speeds
  in certain markets, as well.  The race to deploy increasingly faster
  speeds begs the question of whether users are exhausting the capacity
  that is already available. Previous work has shown that users who are
  already maximizing their usage on a given access link will continue to
  do so when they are migrated to a higher service tier. 

  In a unique controlled experiment involving thousands of Comcast
  subscribers in the same city, we analyzed usage patterns of two groups: a control group
  (105 Mbps) and a randomly selected treatment group that was upgraded
  to 250 Mbps without their knowledge.  We study how users who are
  already on service plans with high downstream throughput respond when
  they are upgraded to a higher service tier without their knowledge, as
  compared to a similar control group. Although subscribers who are
  already using most of their available capacity do not use
  significantly more capacity when they are upgraded to a higher service
  tier, about 50\% of the subscribers who exhibit low traffic demand
  increase the 95th percentile of their daily peak usage by more than 10 MB when
  their service plan is upgraded.  Additionally, we find that
  the median off-peak traffic load increases by about 20\%.
\end{abstract}


