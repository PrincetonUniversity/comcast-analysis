\section{Introduction}
\label{sec:intro}

The Federal Communications Commission (henceforth FCC)



% FCC background and importance
The FCC has been charged with controlling broadband in this country
Their control on ISPs shows that it is an important institute for the future of broadband deployment and use in US
Its responsibility is to increase deployment to rural areas, increase US ranking, get everyone to have broadband connectivity.

% Why it matters, why people should care about it, how many people care about it (eg: retweets on decisions). Do we understand it? (ask Ed about importance and if he would find my paper interesting)

The core of the Internet lies here, service lies here. FCC is the most important commission for the future of Internet access of eyeball American users right now.
Recent Internet trends have changes so much with videos and end-user data.

% Power of FCC: its decisions change the economies of play, some policy papers

Who pays for this? How do we solve issues between content servers, multiple transport providers, and the end-user? FCC is the one setting down the laws. Economies of play change with FCC

% read some policy work here

% Important that their decisions are based on valid measurements and hypothesis.

Thus anything the FCC comes up with must be checked and sanitized. From next steps, to valid metrics to evaluate broadband usage, as well as the definition of broadband. Currently FCC bases its decisions on aggregates throughout this huge country, and divides regions based on rural and urban. It also compares it self to other developed nations. The most recent report as of 2015 Jan ~\cite{fcc2015broadband-report} redefines BB speeds as 25/3 and mentions that rural deployment is not keeping up based on this definition of broadband.

% From FCC report with hypothesis
Section 706 of the Telecommunications Act of 1996, as amended (1996 Act), requires the Commission to determine and report annually on ``whether advanced telecommunications capability is being deployed to all Americans in a reasonable and timely fashion.''

%	- Technical questions and Other types (?)
%	- FCC HYPOTHESIS:
%		- H1:
%		- H2:
%		- H3:
%		- ...

%%%%%%%%%%%%%%%%%%%%%%%%%%%%%%%%%%%%%%%%%%%%%%%%%%%%%%%%%%%%%%%%%%%%%%%%%%%%%%%%%%%%%%%%%%%%
\input{hypothesis}
%%%%%%%%%%%%%%%%%%%%%%%%%%%%%%%%%%%%%%%%%%%%%%%%%%%%%%%%%%%%%%%%%%%%%%%%%%%%%%%%%%%%%%%%%%%%

% Why FCC measurements may not mean jack shit -- needs policy paper references
But (as we see in policy papers) this may not be the right way to go about stuff. Aggregates do not show the right representation of users. And aggregates will not give the right policy model for offering broadband. Now that ISPs are regulated it is important that broadband be offered based on the users demands instead of extracting money out of users who don't need to, or letting users buy highly expensive plans of the ISP when the trouble is that the service they are connecting to is not fast enough.

Also section 706 [Section 706 Advanced Services Inquiry] is shit (based on policy papers) and may not be the right way to go. Although this deals with deployment, hard limits don't make sense. We need to look at sliced measurements from home users, how much data do they use even when given the ability to max out their limits. 

Thus we do Comcast experiment to get data directly from urban city home user gateways on uploads and downloads. We want to comment on high tier users to find out if FCC limits even make sense with usage patterns at all??

%	- To evaluate some of FCC decisions we chose broadband usage data from comcast in an urban city with high tier usage. 
We get data from Comcast to comment on some of these questions in urban single city high tier controlled set. Our analysis sets the baseline and framework for sanitizing the FCC with real measurements, easily available at ISPs, so that future Internet deployment and categorization is not a fuck up.

%Sections: Data + Sanitization + Specifications, Methodololy, Analysis of observations vs FCC Hypothesis
(A roadmap of the paper as follows: ... )