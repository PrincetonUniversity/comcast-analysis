\section{Introduction}
\label{sec:intro}
% para1 : FCC background and importance
% telecom act + responsibility

The large impact of broadband Internet in our daily lives, and it’s rapid increase in providing
services, requires regulation of all involved parties, such as all transit and content providers.
The Federal Communications Commission (FCC) holds the important responsibility of overlooking public
and private sector initiatives to ensure advanced broadband deployment throughout the US. As part of
this responsibility (since the amendment to the Telecommunications Act of 1996 (section 706~\cite{fcc1996telecom-act})
\footnote{This issue is discussed in point $1$, and point $12$ of the Eleventh Broadband Progress Report,
FCC No. 15-10A1, respectively ~\cite{fcc2015progress-report}\label{foot:fcc-issues}}), 
the \FCC has been reporting results of its annual inquiry regarding the
availability and timely deployment of ``advanced telecommunications capability'' to all Americans.

% para2: why its important to us as consumers

If the \FCC determines that \emph{``advanced telecommunications capability''} is not being deployed
to all Americans in a reasonable and timely fashion, the Commission is required to ``take immediate
action to accelerate deployment of such capability by removing barriers to infrastructure investment
and by promoting competition in the telecommunications market.'' \footref{foot:fcc-issues}.
The \FCC accomplishes this role by following three major steps:
(a) \emph{defining a benchmark\footnote{The \FCC uses benchmark to a 
threshold, beyond which the access link will be considered broadband capable.}} 
to evaluate ``advanced telecommunications capability'', (b) \emph{auditing
capabilities} of services offered by Internet Service Providers (ISPs), and (c) \emph{spending the
federal budget} to increase ISP capabilities, and accomplish their defined standards.

% para3: speed benchmarking and broadband deployment
% define advanced telecom capability, speed benchmarking
% FCC reports 2010, 2014, 2015
% our proposal ??
As independent researchers in the measurement community, it is our responsibility
to ensure that the
\FCC defines broadband benchmarks and standards in a sensible manner. Efforts to define such
benchmarks have always been based on aggregated data analysis by the \FCC. The 
source of the FCC's data are mixed across tiers and locations, and collected 
using different methodologies; particularly vantage points in the data were 
treated similarly without studying their actual usage 
requirements~\cite{fcc2015progress-report}. Based on such data, the \FCC made a 
decision to aggressively increase the broadband benchmark threshold from 4 
Mbps/1 Mbps to 25 Mbps/3 Mbps~\cite{fcc-redefine-ieee}\footnote{4Mbps/1Mbps is 
defined as 4Mbps uplink, and 1Mbps downlink capacity.}. This prompted a sharp 
response from ISPs offering technologies that suddenly did not qualify as 
broadband (such as DSL).

We support the move to a higher broadband standard based on the increasing 
share of high quality entertainment traffic in home broadband 
consumption~\cite{sandvine2014report1,sandvine2014report2}, as well as the 
possibility of offering advanced capability (such as telemedicine). But, we 
strongly motivate the Commission to revisit their evaluation criteria in 
reaching more meaningful standards. We also realize that the \FCC requests 
comments on the issues of broadband benchmarking ~\nameref{sec:appendix} in 
future. To this aim, we use a dataset collected purposefully to validate and 
improve the \FCC policy on broadband benchmark.

Our study is based on Comcast’s analytic data collected from home gateways for their customers in
Salt Lake City, Utah. This data consists of byte transfers collected continuously every 15 minutes
from two types of users; control: users that pay and use a high capacity access link (105 Mbps), and
test: users that pay for 105 Mbps but are actually offered a much higher capacity access link
(250Mbps) without their knowledge. This decision was specifically anticipated to enable researchers
to investigate the question: does increase in bandwidth change usage behavior? 
Our analysis shows that not only are we capable of answering the above 
question, but we can also validate and improve more FCC policies on broadband 
benchmark.

% follow directly from the results section
Our contributions in this work are as follows:
\begin{itemize}
\itemsep0em
%%%%%%%%%%%%%% prime time
\item We reaffirm the importance of measuring performance during the 
prime-time hours, and urge the FCC to change and standardize 
the measurement and interpretation of \textbf{prime-time ratio}. 
\re{should we use of ``change'' vs ``revisit''}
%%%%%%%%%%%%% different perspective of utilization
\item We shed light on the user's and ISP's confilicting perspectives 
of \textbf{capacity utilization}, and show that indeed, the user's behavior 
does change when offered a higher capacity link even though the 
overall utilization stops increasing after a certain upper limit.
%%%%%%%%%%%%% user taxonomy and multiple benchmarks
\item We explore a user taxonomy based on \textbf{varying usage behavior} 
within the same tier, and suggest that the FCC adopt multiple benchmarks 
to better characterize broadband availability, deployment, and adoption in the 
US.
\end{itemize}

In this work, after offering details about our data and how it is sanitized in 
[section~\ref{sec:data}], we proceed by describing our methodology and evaluation criteria to study
changes in broadband usage [section~\ref{sec:methodology}]. Next, we analyze spacio-temporal
usage patterns as well as peak usage (during prime-time). Our analysis indicates that for high
capacity access links, increasing capacity does not result in higher utilization, and
motivates the need of multiple benchmarks based on varying usage within the same tier
[section~\ref{sec:results}]. In the last section, we comment on the issues 
raised by \FCC to investigate a better benchmarking solution for the US 
[section~\ref{sec:discussion}].


%%%%%%%%%%%%%% usage behavior
%\item We characterize \textbf{usage behaviors} and show that aggregate 
%network usage seen by the ISP does not show a trough in the middle of the 
%day for high capacity tiers. as observed by aggregate usage across all 
%tiers. This motivates the need %to study usage per tier separately as usage 
%behavior differs with tiers.

%We suggest that the FCC adopt multiple benchmarks to better characterize 
%broadband availability, deployment, and adoption in the US, based on our 
%analysis of varying usage patterns within the same ISP tier.
%We explore a taxonomy of different users based on varying traffic patterns of 
%users within the same capacity tier, and 

%In particular, we believe that there is a need to: (a) define benchmarks based 
%on standards other than broadband speed, such as usage, (b) define 
%multiple benchmarks rather than aggregates based on
%type of usage, and (c) re-examine peak time usage with recent changes in usage 
%patterns to estimate for capacity planning. We also realize that the \FCC 
%requests comments on the issues of broadband benchmarking [section 
%~\ref{sec:issues}] in future. To this aim, we use a dataset collected 
%purposefully to validate and improve the \FCC policy on broadband benchmark.

%we show through our analysis that the \textbf{prime-time ratio
%(ratio of average data rate during prime time hours to non prime time hours)
%does not follow the conventional definition stated by the FCC.
