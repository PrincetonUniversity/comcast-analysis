\section{Introduction}
\label{sec:intro}

% para1 : FCC background and importance
% telecom act + responsibility


The large impact of broadband Internet in our daily lives, and it’s rapid increase in providing
services require regulation on all interested parties, such as all transit and content providers.
The Federal Communications Commission (FCC) holds the important responsibility of overlooking public
and private sector initiatives to ensure advanced broadband deployment throughout the US. As part of
this responsibility, the \FCC has been reporting results of its annual inquiry regarding the
availability and timely deployment of ``advanced telecommunications capability'' to all Americans,
since the amendment to the Telecommunications Act of 1996 (section 706~\cite{fcc1996telecom-act})
\footnote{\hypoth{1} ~\cite{fcc2015progress-report}}.

% para2: why its important to us as consumers

If the \FCC determines that \emph{``advanced telecommunications capability''} is not being deployed
to all Americans in a reasonable and timely fashion, the Commission is required to ``take immediate
action to accelerate deployment of such capability by removing barriers to infrastructure investment
and by promoting competition in the telecommunications market.'' \footnote{\hypoth{12}
~\cite{fcc2015progress-report}}. The \FCC accomplish this major role by following three major steps:
(a) \emph{defining a benchmark} of ``advanced telecommunications capability'', (b) \emph{auditing
capabilities} of services offered by Internet Service Providers (ISPs), and (c) \emph{spending the
federal budget} on increasing ISP’s capabilities, to accomplish their defined standards.

% para3: speed benchmarking and broadband deployment
% define advanced telecom capability, speed benchmarking
% FCC reports 2010, 2014, 2015
% our proposal ??
As independent researchers in the measurement community, it is our responsibility to ensure that the
\FCC defines broadband benchmarks and standards in a sensible manner. Efforts to define such
benchmarks have always been based on aggregated data analysis by the \FCC. The source of data are
mixed across tiers and locations and collected using different methodologies; particularly vantage
points in the data were treated similarly without studying their actual usage
requirements~\cite{report2010-2014}. \re{Therefore, FCC’s regulations can be subjected to biased to
unknown factors.} In January 2015, ISPs rallied against FCC~\cite{fcc-redefine-ieee} in response to
it’s decision to suddenly increase the broadband benchmark standard from 4 Mbps/1 Mbps to 25 Mbps/3
Mbps~\cite{}. \todo{check years and standards}. Although we support the move to a higher broadband
standard based on the increasing share of high quality entertainment traffic in home broadband
consumption~\cite{sandvine2014report1,sandvine2014report2}, as well as possibility of offering
advanced capability such as telemedicine, we disagree with the methodology followed by the
Commission in reaching these standards. By declaring fixed broadband a regulated commodity, it is
necessary that in the future, broadband pricing policy take into account usage characteristics.

In particular, we believe that there is a need to: (a) define benchmarks based on standards other
than broadband speed, such as usage, (b) define multiple benchmarks rather than aggregates based on
type of usage, and (c) re-examine peak time usage with recent changes in usage patterns, to estimate
for capacity planning. We also realize that the \FCC requests comments on the issues of broadband
benchmarking [section ~\ref{sec:issues}]. To this aim, we use a dataset collected purposefully to
validate and improve the \FCC policy on broadband benchmark.


Our study is based on Comcast’s analytic data collected from home gateways for their customers in
Salt Lake City, Utah. This data consists of byte transfers collected continuously every 15 minutes
from two types of users; control: users that pay and use a high capacity access link (105 Mbps), and
test: users that pay for 105 Mbps but are actually offered a much higher capacity access link
(250Mbps) without their knowledge. This decision was specifically anticipated to enable researchers
to investigate the question: does increase in bandwidth change usage behavior? We limit our analysis
to a particular high access link ISP and a single city because this avoids any biases in usage
patterns especially during peak time, as well as different pricing models. 
In this work, after offering details about our data and how data is sanitizaed in [section
~\ref{sec:data}], we proceed by analyzing spacio-temporal usage patterns for the usage data
collected from multiple home gateways, as well as peak usage (during prime-time) [section
~\ref{sec:methodology}]. Our analysis indicates that for high capacity access links, increasing
capacity does not result in higher utilization, and motivate the need of multiple benchmarks based
on varying usage within the same tier [section ~\ref{sec:results}]. In the last section, we comment
on the issues raised by \FCC to investigate a better benchmarking solution for the US [section
~\ref{sec:discussion}].


