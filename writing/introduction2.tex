\section{Introduction}
\label{sec:intro}

% para1 : FCC background and importance
% telecom act + responsibility

With the large impact of broadband Internet in daily lives, the high speed of its evolution and capabilities, and the exponential increase in services, it is imperative that all interested parties, such as the transit and content providers, collaborate to satisfy the broadband demands of the user. The Federal Communications Commission (henceforth \FCC) holds the important responsibility of overlooking public and private sector initiatives to ensure advanced broadband deployment throughout the US. As part of this responsibility, the \FCC has been reporting results of its annual inquiry regarding the availability and timely deployment of ``advanced telecommunications capability'' to all Americans, since the amendment to the Telecommunications Act of 1996 (section 706~\cite{fcc1996telecom-act}) \footnote{\hypoth{1} ~\cite{fcc2015progress-report}}.


% para2: why its important to us as consumers

If the \FCC determines that \emph{``advanced telecommunications capability''} is not being deployed to all Americans in a reasonable and timely fashion, the Commission is required to ``take immediate action to accelerate deployment of such capability by removing barriers to infrastructure investment and by promoting competition in the telecommunications market.'' \footnote{\hypoth{12} ~\cite{fcc2015progress-report}}. Thus, the \FCC follows a three step program: (a) \emph{define the benchmark} of ``advanced telecommunications capability'', (b) \emph{audit the capability} of services offered by Internet Service Providers (henceforth ISPs), and (c) \emph{spend the federal budget} on increasing ISP capabilities, to accomplish their defined standards.


% para3: speed benchmarking and broadband deployment
% define advanced telecom capability, speed benchmarking
% FCC reports 2010, 2014, 2015
% our proposal ??
As independent researchers in the measurement community, it is our responsibility to ensure that the \FCC defines broadband benchmarks and standards in a sensible manner. Until recently, efforts to define such benchmarks were based on aggregated data analysis by the \FCC, where all users were treated similarly without studying their actual usage requirements~\cite{report2010-2014}. This resulted in a sudden increase of the benchmark standard from 1 Mbps/200 kbps to 4 Mbps/1 Mbps in 2010~\cite{}, and to 25 Mbps/3 Mbps in 2015~\cite{} \todo{check years and standards}. Although we support the move to a higher broadband standard based on the increasing share of high quality entertainment traffic in home broadband consumption~\cite{sandvine2014report1}~\cite{sandvine2014report2}, as well as possibility of offering advanced capability such as telemedicine, we disagree with the methodology followed by the Commission in reaching these standards. By declaring fixed broadband a regulated commodity, it is necessary that in the future, broadband pricing policy take into account usage characteristics.


In particular, we believe that there is a need to: (a) define benchmarks based on standards other than broadband speed, such as usage, (b) define multiple benchmarks rather than aggregates based on type of usage, and (c) re-examine peak time usage with recent changes in usage patterns, to estimate for capacity planning. We also realize that the \FCC requests comments on the issues of broadband benchmarking [section ~\ref{sec:issues}], and this work presents a first attempt in the direction of supplementing \FCC policy with measured usage data from home users.


% what is wrong with FCC report, what we can do better ???
% there must be a better way/better dataset, instead of comparing average across countries
% Power of FCC: its decisions change the economies of play, some policy papers
% read some policy work here ??
% should this be related work instead of intro??


% para4: contributions
% brief results + discussion
% Methodology + Data + contributions
To analyze usage characteristics we study byte counter data collected by Comcast home gateways for their customers in Salt Lake City, Utah, with a high capacity access link (105 Mbps) [section ~\ref{sec:data}]. Although we limit our analysis to a particular high access link ISP and a single city, this avoids any biases in usage patterns especially during peak time, as well as different pricing models. And with this controlled setting, we motivate the need of multiple benchmarks based on varying usage, as well a need to study peak usage for different types of users [section ~\ref{sec:results}]. Based on our analysis, we comment on the issues raised by \FCC to investigate a better benchmarking solution for the US [section ~\ref{sec:discussion}]. Our contributions in this work are as follows:
\begin{itemize}
\item ???
\end{itemize}



%%%%%%%%%%%%%%%%%%%%%%%%%%%%%%%%%%%%%%%%%%%%%%%%%%%%%%%%%%%%%%%%%%%%%%%%%%%%%%%%%%%%%%%%%%%%
%\input{hypothesis}
%%%%%%%%%%%%%%%%%%%%%%%%%%%%%%%%%%%%%%%%%%%%%%%%%%%%%%%%%%%%%%%%%%%%%%%%%%%%%%%%%%%%%%%%%%%%

% Why FCC measurements may not mean jack shit -- needs policy paper references
%But (as we see in policy papers) this may not be the right way to go about stuff. Aggregates do not show the right representation of users. And aggregates will not give the right policy model for offering broadband. Now that ISPs are regulated it is important that broadband be offered based on the users demands instead of extracting money out of users who don't need to, or letting users buy highly expensive plans of the ISP when the trouble is that the service they are connecting to is not fast enough.

%Also section 706 [Section 706 Advanced Services Inquiry] is shit (based on policy papers) and may not be the right way to go. Although this deals with deployment, hard limits don't make sense. We need to look at sliced measurements from home users, how much data do they use even when given the ability to max out their limits. 

%Thus we do Comcast experiment to get data directly from urban city home user gateways on uploads and downloads. We want to comment on high tier users to find out if FCC limits even make sense with usage patterns at all??

%	- To evaluate some of FCC decisions we chose broadband usage data from comcast in an urban city with high tier usage. 
%We get data from Comcast to comment on some of these questions in urban single city high tier controlled set. Our analysis sets the baseline and framework for sanitizing the FCC with real measurements, easily available at ISPs, so that future Internet deployment and categorization is not a fuck up.

%Sections: Data + Sanitization + Specifications, Methodololy, Analysis of observations vs FCC Hypothesis
%(A roadmap of the paper as follows: ... )