\section*{Appendix}
\label{sec:appendix}

In the FCC report No. 14-113, released on Aug 5, 2014~\cite{fcc2014progress-report},
the Commission asks some relevant questions about broadband usage, and requests comments
from the community to improve its decision making process. We summarize their hypothesis
and comments with regards to speed benchmarking as follows\footnote{Note that the issue
\# corresponds to the paragraph in the Eleventh Broadband Progress Report, FCC No. 
15-10A1 ~\cite{fcc2015progress-report}\label{foot:fcc-issue-numbers}}

\hypoth{13} \emph{Peak Time Usage.} Peak usage is defined as the average data consumed between 7:00 to 11:00 pm on weeknights. The FCC asks if evaluating peak time usage is an efficient metric, or if average usage over a day should be considered instead.
%Should bandwidth requirements for a typical household be assessed during peak Internet usage periods, from 7 pm to 11 pm on weeknights? Is the "peak usage time" an efficient metric, or should the average usage of a household over a day be considered instead? Does establishing a reasonable household usage scenario during peak periods assist the Commission to identify a benchmark?
%We also seek comment on whether it is reasonable under the statute to set a speed benchmark on the basis of “peak usage time.” Is peak usage time an efficient metric? Should we instead consider the average household usage over a 24-hour period or over some other time period, or in some other manner entirely? We recognize that every household is unique, and that the services each household member uses will vary. We seek comment on whether establishing a reasonable household usage scenario during peak periods will assist the Commission in identifying a benchmark that is a necessary component of “advanced telecommunications capability

\hypoth{14} \emph{Broadband Speed Benchmarks.} Recently, the FCC defined the minimum broadband speed to be 25 Mbps for downlink, and 3 Mbps for uplink, anticipating future usage. The questions asked by the Commission are as follows: (1) What is the required broadband speed for moderate usage? (2)How should the Commission forecast future household broadband uses to justify such a benchmark?
%The FCC 2014 Household Bandwidth Scenarios suggests that a 10 Mbps download speed could accommodate a “Moderate Use Household,” including allowing a family of three at peak periods to stream a movie, participate in online education, surf the web, and have a mobile device syncing to its email account
%15. Does 10 Mbps satisfy current demand, especially during peak time? Even assuming that it does, should the benchmark be higher than the minimum necessary to meet existing demand, i.e., should the benchmark be set to accommodate some level of anticipated future demand, particularly if the Commission does not intend to adjust the benchmark annually? Some forecasts of broadband household needs suggest a higher download speed may be necessary. 38 For example, would a significantly higher download speed, such as 15 or 25 Mbps, more accurately fulfill Congress’s intent? How should the Commission forecast future household broadband uses to justify such a benchmark?

\hypoth{16} \emph{Adoption Based Benchmarks.} The FCC's previous experience suggests that broadband benchmarks should be based 70\% adoption rate to encourage providers to increase broadband penetration. This would motivate a benchmark of 1 Mbps, however, the uplink speed benchmark has been set to 3 Mbps. Does this aggressive limit adequately anticipate the increasing use of symmetrical services, such as two-way video calling?
%e seek comment on whether a 1 Mbps upload speed will suffice to meet the requirements set forth in section 706. The FCC 2014 Household Bandwidth Scenarios suggests that a service capable of 1 Mbps upload speed may not accommodate all household types. 39 A “Moderate-Use Household,” for example, may be able to stream a movie, engage in online education, surf the web, and have a mobile device syncing to its email account all at the same time. A “High-Use Household” could have difficulty simultaneously streaming a movie, making a video call, using cloud storage, and have a mobile device syncing to its email account. Even if a consumer is primarily using its broadband for intensive download applications, such as streaming a movie, a consumer’s viewing experience could be affected if the consumer does not have sufficient upload speeds. 40 For purposes of the next report, should the Commission retain or increase the 1 Mbps benchmark? If the Commission continues to rely on 1 Mbps upload, we seek comment on whether we should continue to rely on 768 kbps as a proxy for 1 Mbps upload speed.
%Other data suggest that it might be appropriate for the Commission to increase the upload speed benchmark for purposes of addressing the statutory requirements in section 706. How should we consider the use of symmetrical services, such as two-way video calling; the uploading of media to social networks; and cloud storage? Which do consumers use more – standard or HD video calls? Today, consumers can have a real-time video consultation over their broadband connection at home with doctors many miles away and this type of service may require higher upload speeds. 44 In addition, some states have adopted a 1.5 Mbps upload speed as a benchmark. We thus seek comment on whether a 1 Mbps upload speed is sufficient to meet the goals set forth in section 706. We also seek comment on whether there is a basis for the disparity between download and upload speeds in any speed threshold(s) used by the Commission. For example, if the Commission increases the download speed benchmark, should it also increase the upload speed benchmark?
%20. If we were to set our benchmark based on adoption, the Commission would need to determine what adoption rate would be necessary to ensure the speed benchmark was reasonable. For example, the Bureau assumed a subscription rate of 70 percent in modeling the costs of deploying broadband to rural America, although we note that was in a different context, and represents a modeling assumption rather than a substantive determination. 51 Would a benchmark based on a 70 percent adoption rate ensure that our benchmark is reasonable, attainable, and sustainable? Would a lower or higher benchmark be better and if so, why? If the median household chooses to adopt a speed tier, does that demonstrate that there is sufficient demand for that speed tier to suggest that all consumers should have the option of subscribing to it? What about a more forward-leaning adoption rate, such as 30 percent?

%\hypoth 18: Should the Commission consider the rates at which consumers are adopting particular speeds when setting a speed benchmark? \sg{Based on SBI reports 2013 ~\cite{} only 25\% Americans in rural areas have access to 25 Mbps broadband -- 18 may be irrelevant to us}
%

\hypoth{19} \emph{Popular Speed Tier Benchmarks.} The FCC is considering setting future speed benchmarks based on the fastest speed tier with a substantial customer subscription. Does a higher speed tier suggest that service of that speed is necessary to enable users to originate and receive high-quality voice, data, graphics, and video telecommunications?
%Does it make sense to base the benchmark on the fastest speed tier for which a substantial portion of the consumers subscribe. How should the Commission define ``substantial portion'' and how should we interpret such demand? 
%19. Should the benchmark be based on the fastest speed tier to which a substantial portion of consumers subscribe? How should the Commission define “substantial portion” and how should we interpret such demand? Would using such a metric accurately reflect the market choices and needs of consumers based on the service offerings available to them? Does a particular adoption rate (to be determined) by consumers with access to broadband demonstrate that there is sufficient demand for that speed tier? Does adoption at a certain speed demonstrate or suggest that service of that speed is necessary to enable users to originate and receive high-quality voice, data, graphics, and video telecommunications? How should we account for the fact that higher speed services may not be offered in parts of the country? Should we look exclusively at adoption rates in areas where a given speed has been deployed if we select an adoption-based benchmark?


\hypoth{22} \emph{Other Speed Benchmarks.} Broadband requirements are not uniform throughout the nation. Some users will have significantly greater needs. The FCC is interested to know whether it should opt for multiple benchmarks depending on user scenario, usage, occupation, etc.? 
%\todo{characterize differing usage even in 100 Mbps/250 tier -- a user taxonomy, include Sandvine report taxonomy and show of variance in aggregate users -- we will show extreme variance in the same band of users and motivate a need of new benchmarks instead of speed.} \sg{does this end up motivating a case for non-net neutrality based on low usage vs high usage? did FCC take the wrong decision -- if we could show our data set uses completely different set of sites etc...}
% 21/ Other Approaches to Establishing a Speed Benchmark. We recognize that there are other methods the Commission could use to set a speed benchmark. We also recognize that, if the Commission were to rely on a “typical use” scenario or an adoption-based approach, as the basis for setting a speed benchmark, other tools might be helpful to verify that the selected benchmark is reasonable. We seek comment on other ways we might set, or verify, a speed benchmark.
%22. Multiple Speed Benchmarks. The foregoing discussion focuses on selecting a single speed benchmark. We also seek comment on whether the Commission should consider establishing multiple benchmarks. Multiple benchmarks could improve our ability to assess whether advanced telecommunications capability is being deployed in a reasonable and timely manner by recognizing that broadband requirements are not uniform throughout the nation. For example, while the “typical use” 10 Mbps download speed benchmark described above is intended to satisfy common household broadband demand, some users, such as larger families or teleworkers, will have significantly greater bandwidth needs. 52 We seek comment on whether the Commission should adopt more than one speed benchmark and if so, how we should use the different benchmarks to evaluate whether deployment is occurring in a reasonable and timely manner.
% 23. In particular, we seek comment on whether the Commission should adopt a forward-looking benchmark to ensure that we can accommodate the nation’s more advanced broadband needs as they develop. Should the Commission establish a speed benchmark for schools? Should we establish a speed benchmark for libraries? If so, what would be an appropriate benchmark or benchmarks? If we were to establish a forward-looking benchmark, how should we use it as an assessment tool under the statute?

%\hypoth{28} \emph{Data Usage.} 
% \sg{price of tier increases but comcast usage is same here?}



%\hypoth{12} \emph{Household Bandwidth Scenarios} (Table 2, ~\cite{fcc2015progress-report}). The typical bandwidth a household may need today varies between 4 to 10 Mbps for low to high usage households during peak period. Is this still valid with continuous introduction of new services and connected devices?
%The scenarios are designed to reflect preliminary assessments of wh at we believe may reflect typical applications/services used by a household during the peak usage time of 7 pm to 11 pm weeknights. To refine the analysis, we seek comment on the Table 2 below.Are the broadband applications identified in theFCC 2014 Household BandwidthScenariosreasonable?  Are the broadband estimates accurate?  Are consumers using significantlydifferent broadband applications or services than those captured in Table 2?We note that theFCC2014Household Broadband Guideand theFCC2014 Household Bandwidth Scenariosreflect a wide range ofspeed requirements,and further note thatnetwork capacity would likely need to exceed these amounts tofully utilize these services and applications without substantial buffering, packet loss, and delay


%What we don't do: latency, application usage, mobile speed benchmarks.
