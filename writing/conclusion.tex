\section{Conclusion}\label{sec:conclusion}

Our work aims to study the effect of increasing broadband capacity on traffic
demand as a controlled experiment. We analyze byte usage counters collected
from Comcast's residential gateways in the same city on a 105 Mbps service
tier. A group of subscribers was unknowingly upgraded a 250 Mbps capacity.
We observed that, when compared to the users in the 105 Mbps speed tier,
the 250 Mbps subscribers sent 10\% more traffic, most of which was in 
off-peak hours.

An increase greater than 20~MB is observed for 40\% of the 
subscribers whose demands are between 10 -- 70~MB. These subscribers
had peak-demands lower than, or comparable to, the median peak demand.
This implies that as the service tier changed, the 95th percentile 
traffic demands of subscribers who contribute the least to the total
traffic volumes increases. Furthermore, the increase in demand was not well
balanced throughout a day. Demand increased by 20\% during non-prime-time
hours on weekdays, when the network is not under peak load. During prime-time
hours the increase in usage was under 4\%.

In the future, we plan to follow up by studying the relationship between
traffic demand and service tier
capacity with similar controlled experiments for lower tiers. 
We expect that at lower tiers, the demand during prime-time will increase 
significantly with an unannounced upgrade, but eventually there will be
service capacity beyond which demand during prime-time hours does not
increase substantially with capacity, yet as our results show, there is an 
increase in overall demand. This threshold may play a key role in defining
demand based broadband benchmarks in the future.

Another direction for future research is to study the reason for increased 
demand in non-prime-time periods among the lighter users. We are currently 
gathering data with application information and byte counter reports every 30 s 
to investigate this further.

\paragraph{POLICY IMPLICATIONS?}

Demand increases for lower bandwidth users consistently
Relatively more data is sent in off peak hours 

A service provider considering increase in 
capacity will not invest in offering a service upgrade in such a scenario. The 
ISP perspective is that the highest demanding subscribers will not be utilizing 
the higher capacity during the prime-time, thus the total demand will not 
increase. This question is considered by ISPs when planning capacity upgrades 
in the future, or considering investment in a new technology or region. For 
example, Google Fiber is now expanding to Salt Lake City, from where 
we received our dataset. The analysis of change in user behavior with capacity 
estimates that a low number of users already on the 105 Mbps service tier will 
actually increase their demand beyond the 105 Mbps capacity if they 
migrate to a higher capacity service offered by Google. 

This may 
convince subscribers and policy makers alike that individually, the demand of a 
household is affected by the increase in access link capacity, especially for 
subscribers who did not have much demands. We believe that this is the 
perspective the FCC takes when considering deployment and adoption of 
broadband services. Essentially, if any change is observed in demand due to an 
upgraded service, the FCC may interpret that as \emph{adoption} of the higher 
available tier.
