\section{Conclusion}\label{sec:conclusion}

Our work aims to study the effect of increasing broadband capacity on traffic
demand as a controlled experiment. We analyze byte usage counters collected
from Comcast's residential gateways in the same city on a 105 Mbps service
tier. A group of subscribers was unknowingly upgraded a 250 Mbps capacity.
We observed that, when compared to the users in the 105 Mbps speed tier,
the 250 Mbps subscribers sent \todo{X}\% more traffic, most of which was in 
off-peak hours.

For the 50 percent users contributing the most to the total traffic, the 95th
percentile traffic demand over a three-month period did not change when the
speed tier increases. But for the 50 percent subscribers that have the lowest
traffic demand in the 105 Mbps tier, the 95th percentile peak demand increases
substantially. Furthermore, the increase in demand was not well balanced 
throughout
a day. Demand increased two-fold during non-prime time hours when the network is 
not
under peak load. During prime-time hours the increase was negligible.

As the heaviest users contribute to most of the total demand, and their demand 
does not increase substantially during prime-time hours, it suggests that the 
ISP is not a bottleneck to the total network traffic. However, the traffic 
demand
per subscriber increases twofold for households with low demand during 
non-prime-time
hours due to the service upgrade.

In conclusion, service upgrades to high speed tiers do increase demand during 
off-peak
hours. And the demand for users utilizing most of the capacity does not increase
as much as for those who are underusing the link. 

In the future, we plan to follow
up by studying the relationship between traffic demand and service tier
capacity with similar controlled experiments for lower tiers. 
We expect that at lower tiers, the demand during prime-time will increase 
significantly with an unannounced upgrade, but eventually there will be
service capacity beyond which demand during prime-time hours does not
increase substantially with capacity, yet as our results show, there is an 
increase in overall demand. This threshold may play a key role in defining
demand based broadband benchmarks in the future.

Another direction for future research is to study the reason for increased 
demand in non-prime-time periods among the lighter users. We are currently 
gathering data with application information and byte counter reports every 30 s 
to investigate this further.