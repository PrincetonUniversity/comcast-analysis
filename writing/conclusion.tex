\section{Conclusion}\label{sec:conclusion}

For heavy hitters, the highest using 50 percent of the users, the 95th 
percentile utilization over a three-month period did not change when the speed 
tier increases. But for lighter subscribers, the 50 percent with a lower peak 
demand, the 95th percentile peak usage increases substantially.


Overall data rate and average utilization are similar
Heavy users who utilize most of the link capacity do
not increase their demand
Average peak ratio doesn’t rise for majority of
subscribers
Prime time seems to be 8p-12a for both control
group and treatment
This suggests that (1) the series are similar and (2) the
ISP is not the bottleneck for heavy users.


For time slots with low data rate over 15 min, peak
demand per subscriber is higher in the test set
Median data rate per subscriber during off-peak
hours also increases
There are a few outliers that have very high demands
(high utilization) that increase in the treatment.
% maybe due to background data, or consistent increase in short flow usage

This suggests that increasing ISP capacity (tier)
affects demand for very high speed tiers, even though
the link is not fully utilized by the affected subscribers.