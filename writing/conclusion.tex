\section{Conclusion}\label{sec:conclusion}

\sg{when compared to the users in the 105 Mbps speed tier, the 250 Mbps 
subscribers sent X\% more traffic, most of which was in off-peak hours. -- to 
make this claim plot the ratio of total usage (scaled by subscribers) in 
treatment and controlled instead of talking about data rates}

For the heaviest 50 percent of the users, the 95th percentile utilization over a 
three-month period did not change when the speed tier increases. But for lighter 
subscribers, the 50 percent with a lower peak demand, the 95th percentile peak 
usage increases substantially.

As the heaviest users contribute to most of the total demand, and their demand 
does not increase substantially during prime-time hours, it suggests that the 
ISP is not a bottleneck to the total demand.

However, if we count the contribution of each user equally, the traffic demand 
per subscriber increases twofold for households with low demand 
during non-prime-time hours due to the service upgrade.

In conclusion, increasing ISP capacity (tier) affects demand for very high 
speed tiers, even though the link is not fully utilized by the 
affected subscribers and the peak traffic demand does not increase 
significantly. The heavy users show different trends compared to the lighter 
ones.

The difference in user behavior motivates a need to study 
traffic demand further so that broadband benchmarks can be redefined in terms 
of the needs of a subscriber. Similar studies in other regions and service 
tiers would also encourage ISPs to offer broadband services if a substantial 
number of customers appear bottlenecked at lower tiers.

One such future work is to study the relationship between traffic demand 
and service tier capacity with similar controlled experiments for lower tiers. 
We expect that at lower tiers, the demand during prime-time will increase 
significantly with an unannounced upgrade, but there will be service capacity 
beyond which demand during prime-time hours does not increase substantially with
capacity. This limit might be a natural point to define broadband benchmarks in  
the future.

Another direction for future research is to study the reason for increased 
demand in non-prime-time periods among the lighter users. We are currently 
gathering data with application information and byte counter reports every 30 s 
to investigate this further.

