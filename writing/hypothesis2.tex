\section{Hypothesis}
\label{hypothesis}

% Background
Report: (11) In the next Report, we will further explore how to incorporate mobile and satellite in our analysis given consumers want access to a high-speed service that is reliable, affordable, and of consistent quality. But they also want broadband on the go.

1. annual inquiry regarding the availability of “advanced telecommunications capability” to all Americans and to determine whether such capability is being deployed to all Americans in a reasonable and timely fashion.

3. "must enable Americans “to originate and receive high-quality voice, data, graphics, and video telecommunications": not supported by 4 Mbps/1 Mbps. The average household has more than 2.5 people, and for family households, the average household size is as high as 4.3. “Advanced telecommunications capability” requires access to actual download speeds of at least 25 Mbps and actual upload speeds of at least 3 Mbps (25 Mbps/3 Mbps ).

% efforts to deploy broadband

% advanced telecom capability - definition?
Section III. DEFINING “ADVANCED TELECOMMUNICATIONS CAPABILITY”
 
% is this definition even sensible
19-22. Advanced telecommunications capability is defined as “high-speed, switched, broadband telecommunications capability that enables users to originate and receive high-quality voice, data, graphics, and video telecommunications using any technology.” Because it is an evolving standard, and there is no single standard for what should qualify as advanced service, the Commission must exercise discretion when it conducts its annual inquiry. Given this, we adopt an approach that is designed to place America at the forefront of broadband offerings and ensure that all Americans, wherever they live, have access to the extensive and ever-expanding offerings available today or on the near horizon.
%The term “advanced” takes on context when we also consider a global view and how other countries define broadband or identify goals concerning advanced telecommunications services. The European Union considers below 30 Mbps to be “basic broadband,” and download connections between 30 Mbps and 100 Mbps to be “fast broadband.

% Maybe multiple standards + options is the way to go
23. NCTA recommended that the Commission “should use multiple benchmarks in performing its section 706 analysis rather than just one” but the FCC chose not to for this time for clarity.
%because doing so would “better reflect the multi-faceted nature of today’s broadband marketplace, which features a wide variety of technologies and services that are able to cater to the varying needs of American consumers.” Others in the record disagree, stating, one standard sends a clear and easily understood message to all parties



III A. Relevant Factors

24. latency, consistency of speeds reached, usage allowances, security.
Current report: only fixed broadband + data for latency is limited, so rely mainly on speeds for now. We anticipate assessing the “totality of the circumstances” in the next Report by looking more robustly at other factors, such as usage allowances and price, latency, whether service at the relevant speed is available on a consistent and reliable basis, and whether the network is secure, which can be as important as speed in determining whether service is available.

III B. Speed Factor

26. The Commission has recognized that the benchmark must be periodically reassessed in light of market offerings and consumer demand. 2010 recommended every 4 years, which turned out to be too slow. Also, schools updated to short term 100 Mbps per 1000 students, and long term 1 Gbps per 1000.

% “advanced,” “high-speed,” and “high-quality”—terms that Congress left to the agency to define—by examining trends in providers’ speed offerings, what technical speeds are required to use various common applications, and data regarding what speeds consumers are adopting when they have the option to purchase various speeds.

III B 1. 28. Providers are offering higher speeds than ever before, and, as discussed below, consumers are adopting them where they are available. Industry-wide, companies are asserting that a minimum of 25 Mbps downstream is required to take advantage of the services widely offered and used today.
% III B 2. 30. Sandvine reports regarding requirements for video, telemedicine, distance learning, higher video quality

III B 3. Consumers Adopt Higher Speeds When They Have the Option

III B 4. “Advanced Telecommunications Capability” Requires 25 Mbps/3 Mbps for
Consumers
III B 5. “Advanced Telecommunications Capability” for Elementary and Secondary Schools
and Classrooms Requires at Least 100 Mbps and, Longer-Term, 1 Gbps

IV E. Americans Without Access to Fixed 25 Mbps/3 Mbps and Mobile 10 Mbps/768 kbps
Service

VI. REMOVING BARRIERS & PROMOTING COMPETITION

% Data source for FCC is not right?
The last two Reports
relied on the National Broadband Map data, commonly called SBI Data, 41 to the extent the Commission
found the data to be sufficiently reliable. 42 In these Reports, the Commission found that the data
regarding mobile and satellite were not sufficiently reliable, among other reasons, for purposes of its
determination pursuant to section 706(b). 43 The Commission thus based its determination on availability
of fixed broadband service and included an assessment of a variety of factors indicative of broadband
availability including physical deployment and also broadband price, quality, and adoption by
consumers.

4. Recent data show that approximately 55 million Americans (17 percent) live in areas unserved by fixed 25 Mbps/3 Mbps broadband or higher service, and that gap closed only by three percentage points in the last year. We therefore conclude that broadband is not being deployed to all Americans in a reasonable and timely fashion. 5. A digital divide persists.

7. Examine broadband adoption—some reasons why Americans choose not to adopt broadband could reflect factors that are relevant to its “availability,” such as price and quality. 

8. United States may lag behind a number of other developed countries with regard to some broadband metrics. 10. factors of availability, including usage allowances and price, latency, whether service at the relevant speed is available on a consistent and reliable basis, and whether the network is secure

12. Commission “take immediate action to accelerate deployment of such capability by removing barriers to infrastructure investment and by promoting competition in the telecommunications market.

