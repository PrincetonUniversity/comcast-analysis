\section{Hypothesis}
\label{hypothesis}

Report: (11) In the next Report, we will further explore how to incorporate mobile and satellite in our analysis given consumers want access to a high-speed service that is reliable, affordable, and of consistent quality. But they also want broadband on the go.

1. annual inquiry regarding the availability of “advanced telecommunications capability” to all Americans and to determine whether such capability is being deployed to all Americans in a reasonable and timely fashion.

3. "must enable Americans “to originate and receive high-quality voice, data, graphics, and video telecommunications": not supported by 4 Mbps/1 Mbps. The average household has more than 2.5 people, and for family households, the average household size is as high as 4.3. “Advanced telecommunications capability” requires access to actual download speeds of at least 25 Mbps and actual upload speeds of at least 3 Mbps (25 Mbps/3 Mbps ).

4. Recent data show that approximately 55 million Americans (17 percent) live in areas unserved by fixed 25 Mbps/3 Mbps broadband or higher service, and that gap closed only by three percentage points in the last year. We therefore conclude that broadband is not being deployed to all Americans in a reasonable and timely fashion. 5. A digital divide persists.

7. Examine broadband adoption—some reasons why Americans choose not to adopt broadband could reflect factors that are relevant to its “availability,” such as price and quality. 

8. United States may lag behind a number of other developed countries with regard to some broadband metrics. 10. factors of availability, including usage allowances and price, latency, whether service at the relevant speed is available on a consistent and reliable basis, and whether the network is secure

12. Commission “take immediate action to accelerate deployment of such capability by removing barriers to infrastructure investment and by promoting competition in the telecommunications market.

